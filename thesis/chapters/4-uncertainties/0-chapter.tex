\chapter{Evaluating of Uncertainties}
\label{ch:Evaluation}

Uncertainty estimation forms the bedrock of any experimental research, including the search for the ttH process in our
study. It serves as a quantifier of the confidence in our results, ensuring that our findings are not merely the
consequence of random fluctuations or approximations made during analysis. Particularly, we focus on the uncertainties
associated with our estimate of the significance of the ttH signal. Significance, in the context of our analysis, is a
measure of how confidently we can claim the presence of the ttH process amidst the background noise. As such, accurately
estimating its uncertainty is crucial as it sets the boundary between discovery and mere chance. This aspect is
especially vital when claiming the observation of new physics, as we seek to minimize the possibility of a false
discovery. In the following sections, we delve deeper into the sources of uncertainties that we account for in our
study: statistical uncertainties, which arise due to limited data, and systematic uncertainties, resulting from
potential biases and approximations in our experimental setup and data analysis.

\section{Statistical Uncertainties}

Statistical uncertainties emerge from the inherently stochastic nature of the data collection process, particularly due
to the limited size of our data sample. This type of uncertainty, which decreases as we accumulate more data, is tied to
our estimate of the significance of the ttH signal.

The significance in our analysis is closely related to the $\mu$ parameter, or the median signal strength. This
parameter scales the expected number of signal events such that the sum of the scaled signal events and background
events matches the observed data: $n_{\text{background}} + \mu \cdot n_{\text{signal}} = n_{\text{data}}$.

However, the actual number of observed events ($n_{\text{data}}$) is not known at the time of model training and
validation. To handle this, we perform what's known as an "Asimov fit". In an Asimov fit, we assume that the actual data
perfectly follows our model prediction. In other words, we operate under the assumption that $\mu$ is 1 - that is, the
observed data is a perfect mixture of background and signal events as predicted by our model. While this assumption may
not hold in reality, the Asimov fit allows us to gain crucial insights into the behavior of our model and estimate
uncertainties related to the significance of our signal.

By performing the Asimov fit, we can gauge the spread in possible values for our significance, indicating the
statistical uncertainty. This aids in understanding the reliability of our results and the range within which the true
value of our signal's significance is likely to fall with a 95\% confidence level.

With the established method, we perform our analysis to estimate the statistical uncertainties associated with our
significance. The estimated bounds of the median signal strength $\mu$ from the Asimov fit are found to be:

$$
    \mu = 1 + \placeholder{XXX} - \placeholder{XXX}
$$

This result gives us a measure of the precision of our significance estimate.
Alternatively, to achieve the desired 95\% confidence level, we would need to multiply the cross section by a factor of
\placeholder{XXX}.

\subsection{Systematic Uncertainties}

Systematic uncertainties in particle physics originate from a wide range of sources, with several pertinent to our
analysis:

Luminosity Uncertainty: The luminosity of the accelerator, or the number of particles within a beam per unit area, is a
fundamental parameter in any experiment. Uncertainties in the measurement of luminosity can translate into uncertainties
in the overall scale of the data.

Electron and Muon Uncertainty: The identification, reconstruction, and isolation of electrons and muons can have
associated uncertainties. Differences in efficiencies between data and simulation can result in systematic errors.

Next Leading Order (NLO) Uncertainty: Predictions of the rates for various processes are typically calculated to leading
order (LO) or next-to-leading order (NLO) in perturbation theory. The precision of these predictions is limited by the
order to which they are calculated, with higher-order terms introducing potential systematic uncertainties.

Final State Radiation (FSR) and Initial State Radiation (ISR) Uncertainties: These uncertainties are associated with the
additional emission of photons from the initial or final state particles. While these effects are included in
simulations, they are based on theoretical models and can have associated uncertainties.

When these systematic uncertainties are combined with the statistical ones, we perform a fit to estimate the bounds on
the signal strength $\mu$. The bounds from this combined fit are found to be:

$$
    \mu = 1 + \placeholder{XXX} - \placeholder{XXX}
$$

This can be translated into the required increase in cross-section to reach the 95\% confidence
level, yielding a factor of \placeholder{XXX}.