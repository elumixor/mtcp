\chapter*{Conclusions}
\addcontentsline{toc}{chapter}{Conclusions}
\label{ch:conclusions}

In this thesis, we have developed and employed advanced machine learning techniques to improve the identification of the
\tth process. The results of this work advance upon previous efforts in the field, demonstrating the potential of
machine learning to enhance the identification of the \tth process.

This research explored the usage of a transformer architecture, adapted for the tabular data (\gls{ftt}),
particularly in the domain of particle physics and event selection.

Additionally, the thesis investigated the impact of dropping the selection cuts and training on the extended training
set. Combined with the large \gls{nn} models, this approach yielded a significant increase in performance, showcasing
the importance of a substantial and diverse dataset for model training in particle physics.

The evaluation of systematic uncertainties was a new step in the \tth analysis, the ranking of the uncertainties was
performed, and the measurement precision on the median strength parameter $\mu$ was estimated.

% Moreover, we have addressed a flaw in the previous analysis where event weights were not used in the computation of the
% \gls{roc} curves.

The binary and multi-class prediction problems were investigated, and it was found that differentiating between all the
classes yielded better results. Furthermore, the two-phase training process is proposed, where the model is initially
trained on the multi-class prediction problem and subsequently switched to binary classification.

The measurement precision on the median signal strength parameter $\mu$ was estimated as ${\mu = 1 +0.49\,/ -0.44}$ with
statistical uncertainties only, and ${\mu = 1 + 0.80\,/- 0.51}$ when systematic uncertainties are also taken into
account. It is noted that statistical and systematic uncertainties contribute about equally to the final result.