\chapter{Conclusions}
\label{ch:Conclusions}

In this thesis, we have developed and employed advanced machine learning techniques to improve the identification of the
ttH process in the ATLAS data at CERN. The results of this work advance upon previous efforts by Severin König and Jan
Presperín \cite{severin, jan}, who established an effective baseline using feed-forward neural networks and grid search
optimization techniques.

This research demonstrates the capacity of more complex machine learning models, such as residual neural networks
(ResNets) and Feature Tokenizer - Transformers (FT-Transformers), to enhance the classification performance.
Specifically, these architectures provided a significant performance boost when trained on an extended training dataset,
showcasing the importance of a substantial and diverse dataset for model training in particle physics.

A significant innovation in this thesis was the application of systematic uncertainties to the ttH analysis, introducing
a higher level of complexity to the analysis. Even though these uncertainties pose a considerable challenge, we showed
that they can be effectively managed with the use of the TRExFitter program. This additional step in the ttH analysis
underscores the multifaceted nature of machine learning applications in particle physics, incorporating not only
statistical, but also systematic uncertainties.

Moreover, we have addressed a flaw in the previous analysis where event weights were not used in the computation of the
Receiver Operating Characteristic (ROC) curves. The inclusion of event weights, which scale the distribution of the
simulated dataset to match the real data, proved essential in evaluating classifier performance and accuracy.

The binary and multi-class prediction problems were investigated, and it was found that differentiating between all the
classes yielded better results. This insight further supports the need for granular and precise classification methods
in ttH identification.

Moving forward, this research opens several avenues for future exploration. Fine-tuning the models by initially training
on a multi-class prediction problem and subsequently transitioning to binary classification could potentially enhance
performance even further. Further research could also explore other regularization techniques and model architectures to
continue advancing the state-of-the-art in ttH process identification.

In conclusion, this work represents a significant contribution to the field, enhancing the identification of the ttH
process and offering novel approaches and techniques that could be instrumental in future research in particle physics.
As the field continues to evolve, these methodologies provide a robust foundation upon which to build and explore.