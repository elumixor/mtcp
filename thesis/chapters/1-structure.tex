\chapter*{Structure of the thesis}
\addcontentsline{toc}{chapter}{Structure of the thesis}

The thesis is structured as follows:

We begin with a short introduction to the fundamentals of particle physics and the \gls{sm} in the
\autoref{ch:Introduction}, explaining key terms such as Feynman diagrams, branching factors, particle
interactions, and decays.

Thereafter, we provide a brief overview of the \gls{lhc} and the \gls{atlas} detector, enabling readers to comprehend
their operational mechanics, data collection techniques, and the subsequent analysis involved.

We then proceed to explain the process of interest - the production of a top-antitop pair and a Higgs boson from two
gluons in a proton-proton collision (\tth), which constitutes the central focus of our research - the separation of \tth
events from other background processes.

This is followed by an in-depth discussion on the concept of significance in particle physics, covering its theoretical
underpinnings and practical implications in event selection.

\autoref{ch:Methodology} outlines the machine learning techniques and strategies utilized, including
problem formulation, details on the \gls{mc} simulation, performance evaluation metrics.

In \autoref{ch:Evaluation}, a comprehensive analysis of the statistical and systematic uncertainties is
presented.

Finally, in \autoref{ch:Conclusions} we conclude by summarizing our findings, discussing the implications
of the research, and providing suggestions for future studies.

For further insights, the \textbf{\nameref{ch:Appendix}} offer additional information about preselection, used features, and
training large \glspl{nn} efficiently.