\section*{Estimation of Uncertainties}

Accurate estimation of uncertainties is crucial in particle physics analyses to quantify the reliability and robustness
of predictions. In the context of our study on the \ttH signal and background separation using deep learning in the \lss
channel, we address two types of uncertainties: statistical and systematic uncertainties.

Statistical uncertainties arise due to the limited size of the dataset used for training and analysis. These
uncertainties stem from the inherent statistical fluctuations in the data and impact the precision of the predicted
results. To estimate statistical uncertainties, we employ statistical methods such as bootstrapping, resampling, or
Monte Carlo techniques. By generating multiple random datasets or variations of the training set, we can evaluate the
statistical fluctuations in the predictions and obtain uncertainty estimates.

Systematic uncertainties, on the other hand, arise from various sources of systematic effects that can affect the
measurements and predictions. These effects can arise from imperfect knowledge of detector response, uncertainties in
theoretical models, calibration procedures, and other experimental factors. Estimating systematic uncertainties requires
careful evaluation and modeling of these sources of uncertainty. This typically involves systematic studies, such as
varying detector parameters, model parameters, or considering different theoretical scenarios. By quantifying the impact
of these variations on the predictions, we can assess the associated systematic uncertainties.

In our work, we carefully account for both statistical and systematic uncertainties in the \ttH signal and background
separation. For statistical uncertainties, we employ resampling techniques to generate multiple training and evaluation
sets, allowing us to assess the variations in the predicted outcomes. We use statistical measures, such as confidence
intervals or standard deviations, to quantify the statistical uncertainties.

To estimate systematic uncertainties, we conduct comprehensive studies of various sources of systematic effects. These
include variations in detector response, model parameters, theoretical assumptions, and other relevant factors. By
analyzing the impact of these variations on the predictions, we can quantify the systematic uncertainties and
incorporate them into our final results. It is important to note that systematic uncertainties require careful
consideration and continuous refinement to ensure accurate and reliable assessments.

By evaluating both statistical and systematic uncertainties, we provide a comprehensive understanding of the reliability
and limitations of our predictions in the \ttH signal and background separation. This allows us to quantify the overall
uncertainty associated with our results and to interpret the significance of any observed effects or deviations from
expectations.