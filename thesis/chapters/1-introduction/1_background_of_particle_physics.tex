\section*{Background of Particle Physics}


Particle physics is a branch of physics that aims to explore the fundamental building blocks of matter and the
fundamental forces governing their interactions. At its core, particle physics seeks to unravel the mysteries of the
universe by studying the smallest constituents of matter and the fundamental processes that shape our physical reality.
By delving into the realm of subatomic particles, particle physicists strive to deepen our understanding of the
fundamental laws of nature.

Particle physics research encompasses a wide range of experiments conducted in laboratories such as the \gls{lhc} at
\gls{cern} and other international collaborations. These experiments involve the acceleration and collision of particles
at incredibly high energies, allowing scientists to probe the nature of matter at unprecedented scales.

The standard model of particle physics, which has been remarkably successful in describing the known particles and their
interactions, forms the foundation of our current understanding. This model incorporates the electromagnetic, weak, and
strong nuclear forces, and provides a framework for classifying particles into fundamental categories, such as quarks,
leptons, and gauge bosons.

However, despite its achievements, the standard model leaves some fundamental questions unanswered. For instance, it
does not account for the existence of dark matter, explain the asymmetry between matter and antimatter in the universe,
or unify all fundamental forces into a single theory. These outstanding questions motivate ongoing research efforts,
pushing the boundaries of particle physics and inspiring new approaches to unravel these mysteries.

In this context, the study of the \ttH (\figref{feyn_tth}) process, involving the production of a top
quark-antiquark pair accompanied by the production of a Higgs boson, holds particular significance. The \ttH process is
a rare and valuable process that provides insight into the interaction between the top quark, the heaviest known
elementary particle, and the Higgs boson, the particle associated with the mechanism that endows other particles with
mass. Understanding the properties and behavior of the \ttH process is crucial for testing the standard model, probing
the Higgs boson's properties, and potentially uncovering new physics beyond the established framework.

By employing deep learning techniques in the \lss channel, this thesis aims to further optimize the separation of the
\ttH signal from background processes, contributing to the ongoing efforts in particle physics research. The utilization
of deep learning offers a promising avenue for enhancing the analysis of complex datasets, improving the discrimination
power between signal and background events, and potentially uncovering subtle patterns and correlations that were
previously challenging to extract.

By addressing these research objectives, this work seeks to advance our understanding of the \ttH process and pave the
way for more precise measurements, refined analyses, and potential discoveries in the field of particle physics.