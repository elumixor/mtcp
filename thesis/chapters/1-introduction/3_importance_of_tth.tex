\section*{Importance of the \ttH Process in Particle Physics}

The ttH process, involving the production of a top quark-antiquark pair in association with a Higgs boson, holds
paramount importance in particle physics research. Understanding the properties, interactions, and decays of the top
quark and the Higgs boson is crucial for probing the fundamental mechanisms underlying the universe and shedding light
on unresolved questions in physics.

The top quark, being the heaviest known elementary particle, plays a unique role in our understanding of particle
physics. Its large mass suggests a special connection to the mechanism of electroweak symmetry breaking, which is
responsible for giving mass to other particles through the Higgs mechanism. Exploring the top quark's properties, such
as its production and decay modes, its couplings to other particles, and its interactions with the Higgs boson, provides
valuable insights into the fundamental nature of particles and the underlying forces governing their behavior.

On the other hand, the Higgs boson, discovered at the LHC in 2012, is a pivotal particle within the standard model. It
is intimately associated with the mechanism that endows elementary particles with mass. Investigating the properties and
behavior of the Higgs boson is crucial for verifying the predictions of the standard model, studying the dynamics of the
electroweak symmetry breaking, and potentially uncovering new physics phenomena beyond the standard model.

The ttH process acts as a bridge between the top quark and the Higgs boson, offering a unique opportunity to study their
interactions. It provides a direct measurement of the coupling strength between the top quark and the Higgs boson, which
is a crucial parameter in the standard model. Precise measurements of this coupling can shed light on the mechanism of
mass generation and help constrain the allowed parameter space for new physics models. Moreover, the ttH process serves
as a testing ground for the validity of the standard model predictions and as a gateway to potential discoveries of new
physics phenomena, such as anomalous couplings or deviations from expected behavior.

However, the ttH process is challenging to study due to its low production rate compared to background processes. The
separation of ttH signal events from background events, characterized by similar final states and overlapping kinematic
properties, poses a significant experimental and analytical challenge. Enhancing the efficiency of signal extraction and
suppressing background contributions is vital for improving the precision of measurements and maximizing the discovery
potential in this process.

In this thesis, we aim to address the challenges associated with the ttH process by employing deep learning techniques
in the 2lSS1tau channel. By optimizing the signal and background separation, we strive to contribute to the
comprehensive understanding of the top quark, the Higgs boson, and their interactions, and potentially uncover new
insights that could reshape our understanding of particle physics.