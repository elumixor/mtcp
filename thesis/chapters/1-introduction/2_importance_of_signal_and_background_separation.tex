\section*{Importance of Signal and Background Separation}

Accurate separation of the signal (\ttH process) from various background processes is of utmost importance in particle
physics research. In experimental analyses, the goal is to identify and extract the signal events corresponding to the
process of interest, while minimizing the impact of background events that mimic or overshadow the signal. The ability
to effectively separate signal from background is crucial for making precise measurements, testing theoretical
predictions, and potentially discovering new phenomena.

In the context of the \ttH process, the challenges associated with signal and background separation are particularly
notable. The \ttH signal events represent a relatively rare occurrence amidst a sea of background events originating from
other physics processes. The background events may arise from processes such as top quark pair production (\ttbar),
associated production of top quark pairs with other particles (such as W or Z bosons), or various multi-jet and
electroweak processes that can mimic the \ttH signal. The kinematic properties, event topologies, and overlapping
features between the signal and background events pose significant challenges for distinguishing them.

Traditionally, various techniques have been employed to discriminate between the signal and background processes,
relying on kinematic variables, event topologies, and other observables. These techniques often involve sophisticated
statistical methods, multivariate analyses, and the incorporation of domain knowledge. However, with the advent of deep
learning, there has been a growing interest in leveraging its power to improve the separation performance.

Deep learning approaches, particularly utilizing neural networks, offer several advantages in tackling the signal and
background separation problem. They have demonstrated remarkable capabilities in learning complex patterns and
correlations directly from data, without relying heavily on explicit feature engineering. Deep learning models can
automatically extract relevant features and capture non-linear relationships that might be challenging to identify
through conventional methods. By training deep neural networks on large datasets, it becomes possible to enhance the
discrimination power and potentially uncover subtle differences that were previously difficult to exploit.

In the context of the \lss channel, the application of deep learning techniques for signal and background separation
holds significant promise. This channel involves the detection of two same-sign leptons (electrons or muons) and an
additional tau lepton, which provides rich information for distinguishing the \ttH signal from background processes. By
exploiting the distinctive characteristics of signal and background events in this channel, deep learning algorithms can
be trained to optimize the separation, leading to improved analysis sensitivity, enhanced discovery potential, and a
deeper understanding of the underlying physics processes.

In this thesis, we focus on further optimizing the \ttH signal and background separation in the \lss channel using deep
learning techniques. By leveraging the power of deep neural networks and utilizing advanced training strategies, we aim
to enhance the discrimination performance, uncover new discriminative features, and contribute to the advancement of
particle physics research in this field.