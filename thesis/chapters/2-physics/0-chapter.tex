\chapter{Overview of Particle Physics and the Standard Model}

Particle physics, often referred to as \gls{hep}, is the study of the fundamental particles and forces of the universe.
It seeks to understand the smallest constituents of matter and how they interact with each other.

\section{Fundamental Particles}

The most basic entities in particle physics are fundamental particles, which are not composed of any other particles.
They include quarks, leptons, and gauge bosons. Quarks and leptons are matter particles, whereas gauge bosons mediate
the fundamental forces.

Quarks come in six "flavors": up, down, charm, strange, top, and bottom. Up and down quarks make up protons and
neutrons, the constituents of atomic nuclei, while the other four are found in high-energy environments, such as
particle accelerators. Quarks possess a unique property known as "color charge" and are always found in combinations
that form "color neutral" particles.

Leptons also come in six types: electrons, muons, taus, and their corresponding neutrinos. Electrons, muons, and taus
all carry electric charge, whereas neutrinos are neutral.

Gauge bosons are force carriers: the photon mediates the electromagnetic force, W and Z bosons mediate the weak nuclear
force, and gluons mediate the strong nuclear force. The Higgs boson, associated with the Higgs field, is responsible for
giving other particles mass.

\section[Standard Model]{\gls{sm}}

\begin{figure}[htb]
    \centering
    \includegraphics[width=\textwidth]{figures/standard-model.pdf}
    \caption{The particles and forces of the Standard Model.}
    \label{fig:standard-model}
\end{figure}

\gls{sm} is a quantum field theory that provides a comprehensive framework for understanding almost all
of the particle physics we observe in the universe. It incorporates the electromagnetic, weak, and strong nuclear forces
into a single theoretical structure.

\gls{sm} is organized around the principle of gauge invariance, which leads to the prediction of the existence of gauge
bosons. It also incorporates a mechanism known as spontaneous symmetry breaking, mediated by the Higgs field, which
gives particles their mass.

\section{Limitations and Beyond the Standard Model}

Despite its incredible success, the Standard Model has limitations. It does not incorporate gravity, does not provide a
candidate for dark matter, and does not explain the matter-antimatter asymmetry in the universe. Moreover, the mechanism
for neutrino masses, which have been experimentally confirmed, is not entirely clear in the SM.

Many theoretical frameworks have been proposed to extend the SM, including Supersymmetry, String Theory, and various
Grand Unified Theories (GUTs). These theories attempt to resolve some or all of these issues, often predicting new
particles and phenomena that could be observed at high-energy particle accelerators.

\section{\ttH Production in the Standard Model}

\input{figures/feynman/tth.tex}
% ttZ
\begin{figure}[htb]
    \centering
    \begin{tikzpicture}
        \begin{feynhand}
            \vertex[dot] (l) at (-1,0) {};
            \vertex[dot] (r) at (1,0) {};

            \vertex[dot] (g1) at (-3,2);
            \vertex[dot] (g2) at (-3,-2);

            \vertex[dot] (t) at (3,2);
            \vertex[dot] (tbar1) at (2,-1);
            \vertex[dot] (tbar2) at (3,-2);
            \vertex[dot, label=\(Z\)] (z) at (3,0);

            % Gluons merging
            \propag [gluon] (g1) to [edge label=\(g\)] (l);
            \propag [gluon] (g2) to [edge label=\(g\)] (l);
            % Gluon propagation
            \propag [gluon] (l) to [edge label=\(g\)] (r);
            % Production of t and tbar
            \propag [fermion] (r) to [edge label=\(t\)] (t);
            \propag [anti fermion] (r) to (tbar1);
            \propag [anti fermion] (tbar1) to [edge label=\(\bar{t}\)] (tbar2);
            % Production of Z
            \propag [boson] (tbar1) to (z);
        \end{feynhand}
    \end{tikzpicture}
    \caption{Feynman diagram of the \ttZ process.}
    \label{fig:feyn_ttz}
\end{figure}
% ttbar
\begin{figure}[htb]
    \centering
    \begin{tikzpicture}
        \begin{feynhand}
            \vertex[dot] (l) at (-1,0) {};
            \vertex[dot] (r) at (1,0) {};

            \vertex[dot] (g1) at (-3,2);
            \vertex[dot] (g2) at (-3,-2);

            \vertex[dot] (t) at (3,2);
            \vertex[dot] (tbar) at (3,-2);

            % Gluons merging
            \propag [gluon] (g1) to [edge label=\(g\)] (l);
            \propag [gluon] (g2) to [edge label=\(g\)] (l);
            % Gluon propagation
            \propag [gluon] (l) to [edge label=\(g\)] (r);
            % Production of t and tbar
            \propag [fermion] (r) to [edge label=\(t\)] (t);
            \propag [anti fermion] (r) to [edge label=\(\bar{t}\)] (tbar);
        \end{feynhand}
    \end{tikzpicture}
    \caption{Feynman diagram of the \ttbar process.}
    \label{fig:feyn_ttbar}
\end{figure}
% ttW
\begin{figure}[htb]
    \centering
    \begin{tikzpicture}
        \begin{feynhand}
            \vertex[dot] (gl) at (-1,1) {};
            \vertex[dot] (gr) at (0.5,1) {};
            \vertex[dot, label=\(\bar{t}\)] (tbar) at (2,1.5);
            \vertex[dot, label=\(t\)] (t) at (2,0.5);
            \vertex[dot] (dbar) at (-3,1.5);

            \vertex[dot] (d) at (-1,-0.75) {};
            \vertex[dot] (u) at (-3,-1.5);
            \vertex[dot, label=\(W^+\)] (w) at (2,-1.5);

            % Down to gluon
            \propag [fermion] (d) to [edge label=\(d\)] (gl);
            % Up to down
            \propag [fermion] (u) to [edge label=\(u\)] (d);
            % Anti down to gluon
            \propag [anti fermion] (dbar) to [edge label=\(\bar{d}\)] (gl);
            % Gluon to gluon
            \propag [gluon] (gl) to [edge label=\(g\)] (gr);
            % Gluon to top
            \propag [fermion] (gr) to (t);
            % Gluon to topbar
            \propag [anti fermion] (gr) to (tbar);
            % Down to W
            \propag [gluon] (d) to (w);
        \end{feynhand}
    \end{tikzpicture}
    \caption{Feynman diagram of the \ttw process.}
    \label{fig:feyn_ttw}
\end{figure}


In \gls{sm}, a top quark-antiquark pair (\ttbar) can be produced in association with a Higgs boson (H), a process
denoted as ttH production. This interaction happens at particle accelerators, such as the Large Hadron Collider, during
high-energy proton-proton collisions.

The production of ttH is a complex process. Initially, gluons from the protons interact to produce a top quark and a top
antiquark. Simultaneously, either a gluon or a top quark can interact with the Higgs field to generate a Higgs boson.

The Higgs boson, top quark, and top antiquark can then decay into various final states, depending on the possible decay
channels of these particles.

The top quark and top antiquark predominantly decay into a W boson and a bottom quark each, given their large coupling
strength. The W bosons can subsequently decay either into a lepton and a neutrino or a pair of quarks.

The Higgs boson, on the other hand, can decay into various particles according to the branching ratios predicted by the
Standard Model. For instance, it can decay into a pair of bottom quarks, a pair of tau leptons, a pair of W or Z bosons,
a pair of photons, and so forth.

\section{\lss Channel}

In the context of ttH production, various decay channels lead to different detectable final states, each with distinct
signatures. The focus of this thesis is the 2lSS1Tau channel.

The \lss channel refers to the final state in which one of the top quarks decays into a W boson and a bottom quark, with
the W boson subsequently decaying into a lepton (electron or muon) and a neutrino. The other top quark follows the same
decay path. The Higgs boson decays into a pair of tau leptons.

The notation $2l_{SS}$ signifies that the two leptons (from the W boson decays) have the same charge, while
$1\tau_\text{had}$ denotes one hadronically decaying tau lepton.

One of the main challenges in studying this channel is to differentiate the \ttH signal from similar processes leading
to the same final state, known as backgrounds. Common background processes include \ttW, \ttZ, and Drell-Yan events.

These events are difficult to distinguish from the \ttH signal as they produce the same final state particles. However,
certain properties, like the kinematic distributions, can provide subtle differences that can be used to differentiate
between signal and background.

The aim of this research is to optimize the separation between the \ttH signal and its background in the 2lSS1Tau
channel using deep learning techniques. The ability to distinguish the signal from the background more efficiently will
allow for more precise measurements of the \ttH production cross section and the top-Higgs Yukawa coupling, potentially
uncovering new physics beyond the Standard Model.