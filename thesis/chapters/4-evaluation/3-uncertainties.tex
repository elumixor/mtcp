\section{Estimating Statistical and Systematic Uncertainties}

In any particle physics analysis, it is crucial to estimate the uncertainties associated with the results. These
uncertainties provide a measure of the confidence we have in our results and are essential for interpreting the
significance of any observed effects. There are two main types of uncertainties to consider: statistical and systematic.

\subsection{Statistical Uncertainties}

Statistical uncertainties arise from the random nature of the data. They are inherent in any process that involves
random sampling, such as the detection of particles in a collider experiment. The size of the statistical uncertainties
typically decreases as the amount of data increases, following the law of large numbers.

In the context of our analysis, the statistical uncertainties are related to the fluctuations in the number of observed
events. For a Poisson distribution, the standard deviation (which provides a measure of the statistical uncertainty) is
equal to the square root of the mean. Therefore, the statistical uncertainty on the number of observed events can be
estimated as the square root of that number.

In terms of the signal strength $\mu$, the statistical uncertainties can be estimated from the likelihood function. The
1-sigma uncertainties correspond to the values of $\mu$ for which the log-likelihood is 0.5 units above its minimum
(this follows from the properties of the chi-square distribution, which the log-likelihood approximately follows). These
values can be found by scanning the log-likelihood as a function of $\mu$ and finding the points where it crosses the
minimum plus 0.5. This gives the upper and lower limits on $\mu$ at the 1-sigma confidence level.

The uncertainty on the signal strength parameter $\mu$ can be estimated using the profile likelihood method. This method
is based on the likelihood function, which gives the probability of the observed data given the model parameters.

The likelihood function for a Poisson distribution is given by:

\begin{equation}
    L(N_{obs}; \lambda) = \frac{\lambda^{N_{obs}} e^{-\lambda}}{N_{obs}!}
\end{equation}

In the case of the signal+background hypothesis, $\lambda$ is given by $N_{bkg} + \mu N_{sig}$, where $N_{bkg}$ is the expected number of
background events, $N_{sig}$ is the expected number of signal events in the simulation, and $\mu$ is the signal strength.

The profile likelihood function is obtained by fixing the value of $\mu$ and maximizing the likelihood with respect to the
other parameters (in this case, there are no other parameters, but in a more complex analysis there could be). This
gives a function of $\mu$ that reflects the compatibility of different values of $\mu$ with the observed data.

The uncertainty on $\mu$ is then given by the values of $\mu$ for which the profile likelihood function decreases by a certain
amount from its maximum. For a 1-sigma (68.3\% confidence level) uncertainty, this amount is 0.5, corresponding to the
change in the log-likelihood for a 1-sigma change in a normally distributed parameter. For a 2-sigma (95.4\% confidence
level) uncertainty, this amount is 2.

This can be calculated as follows:

\begin{enumerate}
    \item Find the value of $\mu$ that maximizes the likelihood function. This is the maximum likelihood estimate (MLE)
          of $\mu$.
    \item Calculate the profile likelihood function by fixing $\mu$ at various values and maximizing the
          likelihood with respect to
          the other parameters.
    \item Find the values of $\mu$ for which the profile likelihood function decreases by 0.5 (for 1-sigma
          uncertainties) or 2 (for 2-sigma uncertainties) from its maximum. These are the lower and upper limits of the
          uncertainty on $\mu$.
\end{enumerate}

This procedure gives the statistical uncertainty on $\mu$, reflecting the random nature of the data. The systematic
uncertainty, reflecting the limitations of the model, would be estimated by a similar procedure, but varying the model
parameters within their uncertainties.

\subsection{Systematic Uncertainties}

Systematic uncertainties, on the other hand, arise from limitations in our understanding or modeling of the experiment.
They can come from various sources, such as uncertainties in the measurement of the detector response, uncertainties in
the theoretical predictions, or uncertainties in the modeling of the background.

Systematic uncertainties are typically estimated by varying the parameters of the model within their uncertainties and
seeing how much the results change. This requires a detailed understanding of the experiment and the analysis, and often
involves a significant amount of work.

The exact procedure for estimating the systematic uncertainties is beyond the scope of this thesis. However, it is worth
noting that they can have a significant impact on the results and must be carefully considered in any analysis. For a
more detailed discussion of systematic uncertainties and how they are estimated, the reader is referred to the CERN
Statistics Committee's guide on statistical tests, which provides a comprehensive overview of the topic
\footnote{CERN Statistics Committee. Statistical Tests. Available at:\\
    https://statisticalmethods.web.cern.ch/StatisticalMethods/statisticaltests/\#stat-systematics.}.

In conclusion, the estimation of statistical and systematic uncertainties is a crucial aspect of our analysis. It
provides a measure of the confidence we have in our results and is essential for interpreting the significance of any
observed effects.







