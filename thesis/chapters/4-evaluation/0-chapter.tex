\chapter{Evaluating Classifier Performance and Estimating Significance and Uncertainties}

In this chapter, we will delve into the application of our trained neural network classifier to our data set, with the
aim of separating the signal (\ttH process) from the background. We will discuss the process of defining a signal region
based on the classifier scores, and how this region is optimized to maximize the sensitivity of our analysis to the
signal process.

We will then move on to the calculation of the significance of our signal. This will involve comparing the number of
observed events in our signal region to the number of expected background events, and calculating the probability that
any observed excess could have occurred by chance under the null hypothesis. We will present the formulas used for this
calculation, and discuss their assumptions and limitations.

Next, we will discuss the estimation of uncertainties. We will differentiate between statistical uncertainties, which
arise from the random nature of the data and can be reduced by collecting more data, and systematic uncertainties, which
arise from limitations in our understanding or modeling of the experiment. We will present methods for estimating these
uncertainties and discuss how they affect our results.

Finally, we will present the results of our analysis, including the estimated significance and uncertainties, and
discuss their implications. We will also discuss potential improvements and future directions for this work.

This chapter will provide a comprehensive overview of how a trained classifier is applied to real data and how its
performance is evaluated in terms of sensitivity, significance, and uncertainties. It will provide a practical guide to
the statistical methods used in particle physics analyses, and demonstrate their application to our specific case of the
\ttH process.

\section{Defining the Signal, Control, and Validation Regions}

In the context of a particle physics analysis, the definition of regions within the data plays a crucial role. These
regions are typically defined based on the scores output by our trained classifier, in this case, a neural network. The
three primary types of regions we consider are the signal region, control regions, and validation regions.

\subsection{Signal Region}

The signal region is the area in the data where we expect the signal to be most prominent compared to the background.
The goal in defining the signal region is to maximize the sensitivity of our analysis to the signal process, in this
case, the ttH process.

The definition of the signal region involves a trade-off between signal efficiency and background rejection. Signal
efficiency is the fraction of signal events that fall within the signal region. We aim for this to be as high as
possible to capture as much of the signal as we can. Background rejection is the fraction of background events that fall
outside the signal region. We aim for this to be as high as possible to reduce the amount of background noise in our
signal region.

The exact boundaries of the signal region are often determined by examining the distribution of classifier scores for
signal and background events in the training data, and choosing a range of scores that maximizes the
signal-to-background ratio.

In our specific case of the 2lss1tau channel, the selection of the signal region is further constrained by the
characteristics of this channel. The specific kinematic properties of the 2lss1tau channel, such as the number and
energy of the leptons and taus, can be used to further refine the signal region and improve the separation between
signal and background.

\subsection{Control and Validation Regions}

In addition to the signal region, we also define control and validation regions. Control regions are regions of the data
that are dominated by one or more known background processes. These regions are used to estimate the contribution of
these backgrounds to the signal region, and to validate the modeling of these backgrounds in our simulation.

Validation regions are similar to control regions, but are used to validate the entire analysis procedure, including the
classifier. They are typically defined to be orthogonal to the signal region, meaning that they do not overlap with the
signal region and do not contain a significant amount of signal. The agreement between the observed data and the
predicted background in the validation region gives us confidence in the validity of our analysis.

\subsection{Well-Modeling of Variables and Blinding}

A crucial aspect of defining these regions is the well-modeling of the variables used in the analysis. Well-modeling
refers to the requirement that the distributions of these variables in the simulation accurately represent the
distributions in the real data. This is important because any discrepancies between the simulation and the data can lead
to biases in the results.

To ensure the integrity of our analysis and to prevent any potential bias in our results, we employ a blinding
mechanism. This mechanism is based on the ratio of the signal (ttH process) to the background (all other processes).

In our analysis, we calculate the signal-to-background ratio (S/B) for each bin in our data. If this ratio is larger
than 0.15, we do not display the real data for that bin in our plots. This is done to prevent any potential bias that
could arise from adjusting the analysis based on the observed data.

This blinding mechanism is applied during the exploratory phase of the analysis and is only lifted in the final stages,
once all selections and procedures have been finalized and validated. This ensures that our definition of the signal
region, as well as other aspects of the analysis, are not influenced by the observed data, thereby maintaining the
objectivity of our results.

In conclusion, the careful definition of the signal, control, and validation regions, taking into account the
well-modeling of the variables and applying a blinding mechanism, is a crucial aspect of our analysis. This allows us to
maximize the sensitivity to the ttH process, estimate and validate the background contributions, and ensure the validity
of our results.
\section{Calculating Significance}

The significance of a signal in a particle physics analysis quantifies the confidence level of a result or observation.
It is a measure of how unlikely a result is to have occurred by chance, assuming the null hypothesis is true. In our
case, the null hypothesis is that there is no ttH process, and any observed excess of events is due to statistical
fluctuations in the data.

\subsection{Poisson Distribution}

The starting point for calculating the significance is the assumption that the number of events follows a Poisson
distribution. The Poisson distribution is a discrete probability distribution that expresses the probability of a given
number of events occurring in a fixed interval of time or space, if these events occur with a known constant mean rate
and independently of the time since the last event.

The probability mass function of the Poisson distribution is given by:

\begin{equation}
    P(k; \lambda) = \frac{\lambda^k e^{-\lambda}}{k!}
\end{equation}

where k is the number of events, $\lambda$ is the mean number of events, and e is the base of the natural logarithm.

\subsection{Significance Calculation for a Single Bin}

In the simplest case of a single bin, the significance can be calculated using the formula:

\begin{equation}
    Z = \frac{N_{obs} - N_{bkg}}{\sqrt{N_{bkg}}}
\end{equation}

where $N_{obs}$ is the number of observed events, $N_{bkg}$ is the number of expected background events, and
$sqrt{N_{bkg}}$ is the standard deviation of the Poisson distribution assumed for the background. This formula is
derived from the properties of the Poisson distribution and gives the number of standard deviations (or "sigmas") that
the observed number of events is away from the expected number of background events.

\subsection{Significance Calculation for Multiple Bins}

In the case of multiple bins, the calculation of the significance becomes more complex. One common method is to use a
likelihood ratio test. The likelihood L under a given hypothesis is the probability of observing the data given that
hypothesis. For a Poisson distribution, the likelihood of observing $N_{obs}$ events given an expected number $\lambda$
of events is:

\begin{equation}
    L(N_{obs}; \lambda) = \frac{\lambda^{N_{obs}} e^{-\lambda}}{N_{obs}!}
\end{equation}

We calculate the likelihoods under two hypotheses: the background-only hypothesis ($L_{bkg}$), where $\lambda = N_{bkg}$
, and the signal+background hypothesis ($L_{sig+bkg}$), where $\lambda = N_{bkg} + N_{sig}$ and $N_{sig}$ is the
expected number of signal events.

We then form the test statistic q as the ratio of these two likelihoods:

\begin{equation}
    q = -2 \ln\left(\frac{L_{bkg}}{L_{sig+bkg}}\right)
\end{equation}

The distribution of q under the null hypothesis can be approximated by a chi-square distribution, from which the p-value
and Z-score can be calculated. The Z-score gives the significance of the signal.

In conclusion, the calculation of the significance is a crucial step in our analysis that allows us to quantify the
confidence level of our results. It involves the use of statistical methods based on the properties of the Poisson
distribution, and takes into account both the observed data and our expectations for the signal and background.

\subsection{Signal Strength and Maximum Likelihood Estimation}

The signal strength, often denoted by $\mu$, is a crucial parameter in our analysis. It quantifies the magnitude of the
signal in the real data relative to the simulated data. In other words, it is a measure of how much the observed data
deviates from the expectations based on the simulation.

The signal strength is typically estimated using the method of Maximum Likelihood Estimation (MLE). The likelihood $L$
under a given hypothesis is the probability of observing the data given that hypothesis. For a Poisson distribution, the
likelihood of observing $N_{\text{obs}}$ events given an expected number $\lambda$ of events is:

\begin{equation}
    L(N_{\text{obs}}; \lambda) = \frac{\lambda^{N_{\text{obs}}} e^{-\lambda}}{N_{\text{obs}}!}
\end{equation}

In the case of the signal+background hypothesis, $\lambda$ is given by $N_{\text{bkg}} + \mu N_{\text{sig}}$, where
$N_{\text{bkg}}$ is the expected number of background events, $N_{\text{sig}}$ is the expected number of signal events
in the simulation, and $\mu$ is the signal strength.

The MLE of $\mu$ is the value that maximizes the likelihood. It can be found by taking the derivative of the
log-likelihood with respect to $\mu$, setting it equal to zero, and solving for $\mu$. This gives:

\begin{equation}
    \mu_{\text{MLE}} = \frac{N_{\text{obs}} - N_{\text{bkg}}}{N_{\text{sig}}}
\end{equation}

This formula shows that the MLE of the signal strength is the ratio of the observed excess of events over the expected
background to the expected number of signal events in the simulation. It provides a measure of how much the observed
data deviates from the expectations based on the simulation, normalized by the expected signal.

In the case of multiple bins, the MLE of $\mu$ is found by maximizing the product of the likelihoods for each bin. This
can be done numerically using optimization algorithms.

In conclusion, the signal strength provides a measure of the magnitude of the signal in the real data relative to the
simulated data, and is estimated using the method of Maximum Likelihood Estimation. This allows us to quantify the
deviation of the observed data from the expectations based on the simulation, and provides a crucial input to the
calculation of the significance.

\section{Estimating Statistical and Systematic Uncertainties}

In any particle physics analysis, it is crucial to estimate the uncertainties associated with the results. These
uncertainties provide a measure of the confidence we have in our results and are essential for interpreting the
significance of any observed effects. There are two main types of uncertainties to consider: statistical and systematic.

\subsection{Statistical Uncertainties}

Statistical uncertainties arise from the random nature of the data. They are inherent in any process that involves
random sampling, such as the detection of particles in a collider experiment. The size of the statistical uncertainties
typically decreases as the amount of data increases, following the law of large numbers.

In the context of our analysis, the statistical uncertainties are related to the fluctuations in the number of observed
events. For a Poisson distribution, the standard deviation (which provides a measure of the statistical uncertainty) is
equal to the square root of the mean. Therefore, the statistical uncertainty on the number of observed events can be
estimated as the square root of that number.

In terms of the signal strength $\mu$, the statistical uncertainties can be estimated from the likelihood function. The
1-sigma uncertainties correspond to the values of $\mu$ for which the log-likelihood is 0.5 units above its minimum
(this follows from the properties of the chi-square distribution, which the log-likelihood approximately follows). These
values can be found by scanning the log-likelihood as a function of $\mu$ and finding the points where it crosses the
minimum plus 0.5. This gives the upper and lower limits on $\mu$ at the 1-sigma confidence level.

The uncertainty on the signal strength parameter $\mu$ can be estimated using the profile likelihood method. This method
is based on the likelihood function, which gives the probability of the observed data given the model parameters.

The likelihood function for a Poisson distribution is given by:

\begin{equation}
    L(N_{obs}; \lambda) = \frac{\lambda^{N_{obs}} e^{-\lambda}}{N_{obs}!}
\end{equation}

In the case of the signal+background hypothesis, $\lambda$ is given by $N_{bkg} + \mu N_{sig}$, where $N_{bkg}$ is the expected number of
background events, $N_{sig}$ is the expected number of signal events in the simulation, and $\mu$ is the signal strength.

The profile likelihood function is obtained by fixing the value of $\mu$ and maximizing the likelihood with respect to the
other parameters (in this case, there are no other parameters, but in a more complex analysis there could be). This
gives a function of $\mu$ that reflects the compatibility of different values of $\mu$ with the observed data.

The uncertainty on $\mu$ is then given by the values of $\mu$ for which the profile likelihood function decreases by a certain
amount from its maximum. For a 1-sigma (68.3\% confidence level) uncertainty, this amount is 0.5, corresponding to the
change in the log-likelihood for a 1-sigma change in a normally distributed parameter. For a 2-sigma (95.4\% confidence
level) uncertainty, this amount is 2.

This can be calculated as follows:

\begin{enumerate}
    \item Find the value of $\mu$ that maximizes the likelihood function. This is the maximum likelihood estimate (MLE)
          of $\mu$.
    \item Calculate the profile likelihood function by fixing $\mu$ at various values and maximizing the
          likelihood with respect to
          the other parameters.
    \item Find the values of $\mu$ for which the profile likelihood function decreases by 0.5 (for 1-sigma
          uncertainties) or 2 (for 2-sigma uncertainties) from its maximum. These are the lower and upper limits of the
          uncertainty on $\mu$.
\end{enumerate}

This procedure gives the statistical uncertainty on $\mu$, reflecting the random nature of the data. The systematic
uncertainty, reflecting the limitations of the model, would be estimated by a similar procedure, but varying the model
parameters within their uncertainties.

\subsection{Systematic Uncertainties}

Systematic uncertainties, on the other hand, arise from limitations in our understanding or modeling of the experiment.
They can come from various sources, such as uncertainties in the measurement of the detector response, uncertainties in
the theoretical predictions, or uncertainties in the modeling of the background.

Systematic uncertainties are typically estimated by varying the parameters of the model within their uncertainties and
seeing how much the results change. This requires a detailed understanding of the experiment and the analysis, and often
involves a significant amount of work.

The exact procedure for estimating the systematic uncertainties is beyond the scope of this thesis. However, it is worth
noting that they can have a significant impact on the results and must be carefully considered in any analysis. For a
more detailed discussion of systematic uncertainties and how they are estimated, the reader is referred to the CERN
Statistics Committee's guide on statistical tests, which provides a comprehensive overview of the topic
\footnote{CERN Statistics Committee. Statistical Tests. Available at:\\
    https://statisticalmethods.web.cern.ch/StatisticalMethods/statisticaltests/\#stat-systematics.}.

In conclusion, the estimation of statistical and systematic uncertainties is a crucial aspect of our analysis. It
provides a measure of the confidence we have in our results and is essential for interpreting the significance of any
observed effects.







