\section{Defining the Signal, Control, and Validation Regions}

In the context of a particle physics analysis, the definition of regions within the data plays a crucial role. These
regions are typically defined based on the scores output by our trained classifier, in this case, a neural network. The
three primary types of regions we consider are the signal region, control regions, and validation regions.

\subsection{Signal Region}

The signal region is the area in the data where we expect the signal to be most prominent compared to the background.
The goal in defining the signal region is to maximize the sensitivity of our analysis to the signal process, in this
case, the ttH process.

The definition of the signal region involves a trade-off between signal efficiency and background rejection. Signal
efficiency is the fraction of signal events that fall within the signal region. We aim for this to be as high as
possible to capture as much of the signal as we can. Background rejection is the fraction of background events that fall
outside the signal region. We aim for this to be as high as possible to reduce the amount of background noise in our
signal region.

The exact boundaries of the signal region are often determined by examining the distribution of classifier scores for
signal and background events in the training data, and choosing a range of scores that maximizes the
signal-to-background ratio.

In our specific case of the 2lss1tau channel, the selection of the signal region is further constrained by the
characteristics of this channel. The specific kinematic properties of the 2lss1tau channel, such as the number and
energy of the leptons and taus, can be used to further refine the signal region and improve the separation between
signal and background.

\subsection{Control and Validation Regions}

In addition to the signal region, we also define control and validation regions. Control regions are regions of the data
that are dominated by one or more known background processes. These regions are used to estimate the contribution of
these backgrounds to the signal region, and to validate the modeling of these backgrounds in our simulation.

Validation regions are similar to control regions, but are used to validate the entire analysis procedure, including the
classifier. They are typically defined to be orthogonal to the signal region, meaning that they do not overlap with the
signal region and do not contain a significant amount of signal. The agreement between the observed data and the
predicted background in the validation region gives us confidence in the validity of our analysis.

\subsection{Well-Modeling of Variables and Blinding}

A crucial aspect of defining these regions is the well-modeling of the variables used in the analysis. Well-modeling
refers to the requirement that the distributions of these variables in the simulation accurately represent the
distributions in the real data. This is important because any discrepancies between the simulation and the data can lead
to biases in the results.

To ensure the integrity of our analysis and to prevent any potential bias in our results, we employ a blinding
mechanism. This mechanism is based on the ratio of the signal (ttH process) to the background (all other processes).

In our analysis, we calculate the signal-to-background ratio (S/B) for each bin in our data. If this ratio is larger
than 0.15, we do not display the real data for that bin in our plots. This is done to prevent any potential bias that
could arise from adjusting the analysis based on the observed data.

This blinding mechanism is applied during the exploratory phase of the analysis and is only lifted in the final stages,
once all selections and procedures have been finalized and validated. This ensures that our definition of the signal
region, as well as other aspects of the analysis, are not influenced by the observed data, thereby maintaining the
objectivity of our results.

In conclusion, the careful definition of the signal, control, and validation regions, taking into account the
well-modeling of the variables and applying a blinding mechanism, is a crucial aspect of our analysis. This allows us to
maximize the sensitivity to the ttH process, estimate and validate the background contributions, and ensure the validity
of our results.