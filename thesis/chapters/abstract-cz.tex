\chapter*{Abstrakt}
% \addcontentsline{toc}{chapter}{Abstract}

Tento výzkum se soustředí na cíl přesně vybrat události vzniklé interakcí dvou gluonů při srážce protonu s protonem na
urychlovači \acrshort{lhc}, jejímž výsledkem je pár top-antitop a Higgsův boson, což je proces známý jako \tth.

Cílem studie je pomocí dat zaznamenaných detektorem \acrshort{atlas} odlišit tyto události \tth od událostí generovaných
jinými procesy. Za tímto účelem využíváme pro výběr událostí přístup hlubokého učení, konkrétně architekturu
\acrshort{ftt}.

Použití takové metody strojového učení zvyšuje naši schopnost přesně identifikovat události \tth, což vede ke zlepšení
poměru signálu k šumu a statistické významnosti, a tím přispívá k našemu pochopení vlastností Higgsova bosonu. Kromě
zkoumání pokročilejších architektur NN vylepšujeme předchozí práci tím, že zkoumáme použití rozšířené trénovací množiny,
což nám umožňuje výrazně zvýšit trénovací statistiku a dosáhnout tak mnohem lepšího výkonu.

Nedílnou součástí tohoto výzkumu je vyhodnocení statistických i systematických nejistot spojených s tímto procesem
výběru událostí. Zjištění a metodiky prezentované v této práci nabízejí slibný pokrok v oblasti výběru událostí
částicové fyziky a přispívají k pokračujícímu úsilí kolaborace \acrshort{atlas} o zkoumání základních vlastností
vesmíru.

\vspace{3mm}
\noindent
\textbf{Keywords:}
CERN, ATLAS, Higgsův boson, Strojové učení, Neuronové Sítě