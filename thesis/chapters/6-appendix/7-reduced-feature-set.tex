\section{Effect of the reduced feature set}

The process of reducing the total number of features in a classification model is a common technique used to minimize
the systematic uncertainties that may arise from the use of a large number of features. The systematic uncertainties are
a result of the inherent limitations of the data and the model used to analyze it. These uncertainties can arise from a
variety of sources, including the statistical fluctuations in the data, the limitations of the model used to analyze the
data, and the presence of unaccounted-for systematic effects.

To minimize these uncertainties, it is best to have as few features as possible. However, this trade-off comes at the
cost of losing some information, which can lead to a reduced performance of the classifier. Therefore, it is important
to carefully select the features that are most relevant to the classification problem at hand.

In addition to selecting the most relevant features, it is also important to ensure that the selected features are well
modeled. This means that the distributions of the features in the simulated data should agree with the recorded data
well enough. If the distributions do not agree, it can lead to biases in the classification results, which can affect
the accuracy of the model.

To ensure that the selected features are well modeled, it is important to perform a thorough analysis of the data and
the model used to analyze it. This analysis should include a comparison of the distributions of the features in the
simulated data and the recorded data, as well as an evaluation of the systematic uncertainties associated with the
model. By carefully selecting and modeling the features used in the classification model, it is possible to minimize the
systematic uncertainties and improve the accuracy of the model.

We have performed 7 different experiments each with a different feature set. The results are shown in
\autoref{tab:feature_sets}. To leave out the effects of random initialization, we have always started with the same
\gls{nn} using all the features. Then we have removed the input neurons corresponding to the features that we wanted to
leave out. We have also kept the same random seed for each experiment. The results show that the classifier trained on
all the features performs the best. However, the classifier trained on the top 20 most important features performs
almost as well, which makes it suitable for reducing the systematic uncertainty.

\begin{table}[htbp]
    \centering
    \begin{tabular}{lccc}
        \toprule
        Feature Set                                             & Accuracy & $\text{AUC}_\text{Mean}$ & $\text{AUC}_\tth$ \\
        \midrule
        All features                                            & 0.85     & 0.92                     & 0.87              \\
        Only well-modelled features      (34 features)          & 0.81     & 0.89                     & 0.83              \\
        Top 20 most important features                          & 0.83     & 0.91                     & 0.85              \\
        Top 20 best modelled features                           & 0.79     & 0.87                     & 0.81              \\
        Top 20 most important (of the well-modelled)            & 0.82     & 0.90                     & 0.84              \\
        Top 20 with least systematical uncertainty contribution & 0.84     & 0.91                     & 0.86              \\
        Features from the BDT from the Rome's group             & 0.80     & 0.88                     & 0.82              \\
        \bottomrule
    \end{tabular}
    \caption{Comparison of different feature sets}
    \label{tab:feature_sets}
\end{table}