\section{Why Supervised Learning?}
\label{appendix:why-supervised}

In the context of classifying specific particle interactions like \tth events, supervised learning is often favored for
several reasons:

\begin{enumerate}
      \item Labeling Ground Truth: With a well-established theoretical model and simulation techniques, you can generate
            labeled datasets that represent the intermediate and final states of particle collisions. This allows for
            direct training on what you want the model to learn.

      \item Efficiency and Precision: Supervised learning can leverage this labeled data to create precise and targeted
            models for the classification problem at hand. This generally results in a more efficient learning process
            compared to unsupervised methods like clustering, which may not have specific labels to guide the learning.

      \item Model Interpretability: By aligning the model with labeled examples, it's often easier to interpret the
            model's decision process and understand how it correlates the inputs to the desired classification. This can
            be important in scientific contexts where understanding the model's behavior can be as crucial as its
            predictive accuracy. In the case of our task, specifically, a particular use-case would be to determine which
            parts of detectors deserve more attention during improvement, and which one may not be so relevant.

      \item Directly Aligned with the Objective: If the main goal is to classify specific types of events, then using
            supervised learning directly aligns with this objective. It utilizes the available knowledge of the phenomena
            being studied (in this case, particle physics and the standard model) to create a learning paradigm
            specifically designed to recognize those events.

      \item Potential for Fine-Tuning: Supervised models can often be fine-tuned or adjusted with new data or different
            structures to continually improve or adapt to new insights. This iterative refinement aligns well with the
            scientific method of incremental understanding and improvement.

      \item Control Over Error and Bias: In the supervised learning framework, you can often have more control over the
            types of errors and biases that the model may introduce, as you're explicitly defining what constitutes a
            correct classification.

      \item Challenges with Unsupervised Learning: On the contrary, unsupervised learning methods like clustering would
            require determining similarities between events without clear labels, potentially leading to ambiguous or less
            accurate classifications. It might not leverage the rich theoretical knowledge available in the field of
            particle physics.

      \item In conclusion, while unsupervised methods might be useful in exploratory phases or when labeled data is not
            available, the specific nature of ttH event classification, combined with the availability of simulated
            labeled data and theoretical grounding, makes supervised learning a well-suited and likely more effective
            approach.
\end{enumerate}

Clustering, as an unsupervised learning technique, can be applied to the context of our task, but with some significant
caveats. Here's how it might work and the challenges involved:

\begin{enumerate}
      \item Application: Clustering could be used to group collision events based on similarities in their observable
            features, without prior knowledge of labels. By identifying patterns in the data, you might uncover groups
            that correspond to different types of events, such as ttH and others.

      \item Prediction: Once the clusters have been identified, you could theoretically assign labels to those clusters
            based on expert knowledge or additional analysis. These labels could then be used to classify new data.
            However, this is where the challenges and limitations become apparent.

      \item Challenges and Limitations: Lack of Ground Truth: Without labeled data to guide the clustering, there's no
            straightforward way to ensure that the clusters align with the actual underlying physics. The clusters might
            correspond to some other aspects of the data that aren't relevant to the classification task.

      \item Ambiguity: The boundaries between clusters may be ambiguous, leading to uncertainty in how to label the
            clusters. This could result in less accurate classifications.

      \item Model Complexity: Interpreting and validating the clusters might require significant expertise and additional
            modeling, making it a more complex and potentially less reliable approach than supervised learning.

      \item Predictive Power: Without a clear mapping from features to labels (as in supervised learning), the predictive
            power of a clustering-based model might be lower, especially if the clusters do not correspond well to the
            underlying physics.

      \item Not Directly Aligned with the Task: Clustering is generally used for exploratory data analysis and pattern
            recognition rather than direct classification. Adapting it to a classification task like identifying ttH
            events may require significant modification and may not be as efficient or effective as using a method
            specifically designed for classification, such as supervised learning.
\end{enumerate}
