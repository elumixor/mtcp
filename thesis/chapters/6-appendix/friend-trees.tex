\clearpage

\section{Friend Trees}
\label{appendix:friend-trees}

To be able to perform the analysis of the impact of the systematic uncertainties in \trex using the custom classifier's
output variable, one must either inject the variable in all the root files, or create the so-called friend n-tuples,
where each file only contains the classifier's output variable. The latter is the preferred option, as it does not
require modifying or copy the original files (which are $\approx 2.7$ TB in size).

First, one must essentially process all the events with the classifier. However, to make the processing shorter, the
cuts can be applied to all the systematic files first, which reduces the total size to $\approx 40$ GB. Then, the
classifier should be applied to all the events to produce the output variable. For each input file, a new "friend"
file is produced, which contains only the classifier's output variable.

Once the friend trees are produced, \trex should be made aware of them.

To do so, \texttt{UseFriend: TRUE} should be added across all the \texttt{Sample} blocks. Then, in the \texttt{Job} block
definition, \texttt{FriendPath: XXX\_FriendPaths} should be added, with the \texttt{XXX\_FriendPaths} pointing to the
\texttt{nominal} directory of the trees.

Furthermore, for the systematic blocks definitions, where the \texttt{NtuplePathsUp} or \texttt{NtuplePathsDown} are
defined, the \texttt{FriendPathsUp} and \texttt{FriendPathsDown} should be added, respectively.

This enables \trex to access both the original trees containing all the systematics-related information, and
the friend trees containing only the classifier's output variable.

\clearpage