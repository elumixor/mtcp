\chapter{Appendix}
\label{ch:Appendix}


% Suggestion for the  future progress perhaps
% \section{Automated well-modelling estimation}

% This "well enough" usually just means that a person looks at the plots and decides if the agreement is good enough.
% But we propose an automated way of doing this. As the algorithm is basically checking if the number of weighted events
% in each bin is the same for the simulated and recorded data. However, only the bins where the signal to background ratio
% is high enough are considered.

% Each bin thus contributes to the well-modelling score. If the bin is blinded, the contribution is zero. Otherwise, the
% contribution for the bin is:

% $$
%     \text{Bin Contribution} = 1 - \frac{N_\text{Recorded} - N_\text{Simulation}}{\max(N_\text{Recorded}, N_\text{Simulation})}
% $$

% Where $N_\text{Recorded}$ and $N_\text{Simulation}$ are the number of weighted events in the bin for the recorded and
% simulated data respectively.

% The total well-modelling score is then the mean of the contributions for all the bins.


\section{\gls{sr} cut expression}
\label{appendix:cut-expression}

{\scriptsize
    \begin{verbatim}
    custTrigMatch_LooseID_FCLooseIso_DLT
    && (dilep_type && (lep_ID_0*lep_ID_1)>0)
    && ((lep_Pt_0 >= 10e3 && lep_Pt_1 >= 10e3) && (fabs(lep_Eta_0) <= 2.5 && fabs(lep_Eta_1) <= 2.5)
        && ((abs(lep_ID_0) == 13 && lep_isMedium_0 && lep_isolationLoose_VarRad_0 && passPLIVTight_0)
            || ((abs(lep_ID_0) == 11 && lep_isTightLH_0 && lep_isolationLoose_VarRad_0 && passPLIVTight_0
                && lep_ambiguityType_0 == 0 && lep_chargeIDBDTResult_recalc_rel207_tight_0 > 0.7)
                && ((!(!(lep_Mtrktrk_atConvV_CO_0 < 0.1 && lep_Mtrktrk_atConvV_CO_0 >= 0 && lep_RadiusCO_0 > 20)
                    && (lep_Mtrktrk_atPV_CO_0 < 0.1 && lep_Mtrktrk_atPV_CO_0 >= 0)))
                    && !(lep_Mtrktrk_atConvV_CO_0 <0.1 && lep_Mtrktrk_atConvV_CO_0 >= 0 && lep_RadiusCO_0 > 20))))
            && ((abs(lep_ID_1) == 13 && lep_isMedium_1 && lep_isolationLoose_VarRad_1 && passPLIVTight_1)
                || ((abs(lep_ID_1) == 11 && lep_isTightLH_1 && lep_isolationLoose_VarRad_1 && passPLIVTight_1
                    && lep_ambiguityType_1 == 0 && lep_chargeIDBDTResult_recalc_rel207_tight_1 > 0.7)
                    && ((!(!(lep_Mtrktrk_atConvV_CO_1 < 0.1 && lep_Mtrktrk_atConvV_CO_1 >= 0 && lep_RadiusCO_1 > 20)
                        && (lep_Mtrktrk_atPV_CO_1 < 0.1 && lep_Mtrktrk_atPV_CO_1 >= 0)))
                        && !(lep_Mtrktrk_atConvV_CO_1 < 0.1 && lep_Mtrktrk_atConvV_CO_1 >= 0 && lep_RadiusCO_1 > 20)))))
    && nTaus_OR==1
    && nJets_OR_DL1r_85>=1
    && nJets_OR>=4
    && ((dilep_type==2) || abs(Mll01-91.2e3)>10e3)
\end{verbatim}
}

We have kept the cuts the same as \cite{severin}, except for the cut on the \verb|nJets_OR| to \verb|>=4| to keep
consistent definition \gls{sr} definition across the group \todo{refer to the BDT group - how?}.

\section{Yields Plots}
\label{appendix:yields}

\begin{figure}[htb!]
    \centering
    \begin{subfigure}{0.45\textwidth}
        \includegraphics[width=\linewidth]{figures/yields/lep-pt-0.pdf}
        \caption{Distribution of the transverse momentum of the leading lepton.}
    \end{subfigure}\hfill%
    \begin{subfigure}{0.45\textwidth}
        \includegraphics[width=\linewidth]{figures/yields/lep-pt-1.pdf}
        \caption{Distribution of the transverse momentum of the subleading lepton.}
    \end{subfigure}
\end{figure}

\begin{figure}[htb!]
    \centering
    \begin{subfigure}{0.45\textwidth}
        \includegraphics[width=\linewidth]{figures/yields/n-jets.pdf}
        \caption{Distribution of the number of jets.}
    \end{subfigure}\hfill%
    \begin{subfigure}{0.45\textwidth}
        \includegraphics[width=\linewidth]{figures/yields/n-bjets.pdf}
        \caption{Distribution of the number of $b$-jets.}
    \end{subfigure}
\end{figure}

\begin{figure}[htb!]
    \centering
    \begin{subfigure}{0.45\textwidth}
        \includegraphics[width=\linewidth]{figures/yields/tau-width.pdf}
        \caption{Distribution of the $\tau$-jet width.}
    \end{subfigure}\hfill%
\end{figure}
\section{Event Weighting}
\label{appendix:weights}

\begin{align*}
    \text{yearFactor}       & = \begin{cases}
                                    36646.74 & \text{if } \text{RunYear} = 2015 \text{ or } \text{RunYear} = 2016 \\
                                    44630.6  & \text{if } \text{RunYear} = 2017                                   \\
                                    58791.6  & \text{if } \text{RunYear} = 2018
                                \end{cases}                                                                                                                                            \\
    \text{normalizedFactor} & = \frac{\text{yearFactor}}{140068.94}                                                                                                                                                                                      \\
    \text{weights}          & = \text{custTrigSF\_LooseID\_FCLooseIso\_SLTorDLT} \times \text{weight\_pileup} \times \text{jvtSF\_customOR} \times \text{bTagSF\_weight\_DL1r\_85} \times \text{XXX\_VV\_NJET} \times \text{weight\_mc} \times \text{xs} \\
    \text{lepWeights}       & = \text{lep\_SF\_CombinedTight\_0} \times \text{lep\_SF\_CombinedTight\_1} \times \text{lepSF\_PLIV\_Prompt\_0} \times \text{lepSF\_PLIV\_Prompt\_1}                                                                       \\
    \text{result}           & = \frac{\text{normalizedFactor} \times \text{weights} \times \text{lepWeights}}{\text{totalEventsWeighted}}
\end{align*}

\begin{align*}
    \text{condition1} & = \text{mcChannelNumber} = 364250 \text{ or } 364253 \leq \text{mcChannelNumber} \leq 364255 \text{ or } 364283 \leq \text{mcChannelNumber} \leq 364287 \text{ or } 363355 \leq \text{mcChannelNumber} \leq 363360 \text{ or } \\
                      & \phantom{=} \text{mcChannelNumber} \in \{363489, 345705, 345706, 345715, 345718, 345723, 364290, 364289, 364288\}                                                                                                              \\
    \text{value1}     & = \begin{cases}
                              1.0      & \text{if } nJets\_OR = 0    \\
                              0.986980 & \text{if } nJets\_OR = 1    \\
                              0.853062 & \text{if } nJets\_OR = 2    \\
                              0.785437 & \text{if } nJets\_OR = 3    \\
                              0.741692 & \text{if } nJets\_OR = 4    \\
                              0.709992 & \text{if } nJets\_OR = 5    \\
                              0.685452 & \text{if } nJets\_OR = 6    \\
                              0.665613 & \text{if } nJets\_OR \geq 7
                          \end{cases}                                                                                                                                                                                       \\
    \text{value2}     & = 1.0                                                                                                                                                                                                                          \\
    XXX\_VV\_NJET     & = \begin{cases}
                              \text{value1} & \text{if } \text{condition1} \\
                              \text{value2} & \text{otherwise}
                          \end{cases}
\end{align*}

Luminosity: The total number of simulated events often does not correspond to the luminosity of the real data. Events
are therefore weighted to correspond to the correct integrated luminosity. If $L_{data}$ is the integrated luminosity of
the data and $L_{\gls{mc}}$ is the integrated luminosity of the \gls{mc} sample, then the luminosity weight for an event is $w_{L} =
    L_{data}/L_{\gls{mc}}$.

Cross-Section: Different processes have different probabilities (cross-sections) of occurring. The ratio of the
cross-sections in data and simulation $\sigma_{data}/\sigma_{\gls{mc}}$ is used as a weight.

Detector Effects: The detector response is not always perfectly simulated. Therefore, weights are applied to correct for
known discrepancies in detector efficiencies and energy resolutions between the data and the simulation.

Pileup: Pileup refers to additional proton-proton collisions that occur simultaneously with the event of interest.
Pileup can significantly affect the event reconstruction. Pileup weights are used to match the pileup distribution in
the data.

Higher-order corrections: Theoretical predictions often include higher-order corrections, known as K-factors, to account
for processes beyond the leading-order approximation used in the simulation.
\section{Why Supervised Learning?}
\label{appendix:why-supervised}

In the context of classifying specific particle interactions like \tth events, supervised learning is often favored for
several reasons:

\begin{enumerate}
      \item Labeling Ground Truth: With a well-established theoretical model and simulation techniques, you can generate
            labeled datasets that represent the intermediate and final states of particle collisions. This allows for
            direct training on what you want the model to learn.

      \item Efficiency and Precision: Supervised learning can leverage this labeled data to create precise and targeted
            models for the classification problem at hand. This generally results in a more efficient learning process
            compared to unsupervised methods like clustering, which may not have specific labels to guide the learning.

      \item Model Interpretability: By aligning the model with labeled examples, it's often easier to interpret the
            model's decision process and understand how it correlates the inputs to the desired classification. This can
            be important in scientific contexts where understanding the model's behavior can be as crucial as its
            predictive accuracy. In the case of our task, specifically, a particular use-case would be to determine which
            parts of detectors deserve more attention during improvement, and which one may not be so relevant.

      \item Directly Aligned with the Objective: If the main goal is to classify specific types of events, then using
            supervised learning directly aligns with this objective. It utilizes the available knowledge of the phenomena
            being studied (in this case, particle physics and the standard model) to create a learning paradigm
            specifically designed to recognize those events.

      \item Potential for Fine-Tuning: Supervised models can often be fine-tuned or adjusted with new data or different
            structures to continually improve or adapt to new insights. This iterative refinement aligns well with the
            scientific method of incremental understanding and improvement.

      \item Control Over Error and Bias: In the supervised learning framework, you can often have more control over the
            types of errors and biases that the model may introduce, as you're explicitly defining what constitutes a
            correct classification.

      \item Challenges with Unsupervised Learning: On the contrary, unsupervised learning methods like clustering would
            require determining similarities between events without clear labels, potentially leading to ambiguous or less
            accurate classifications. It might not leverage the rich theoretical knowledge available in the field of
            particle physics.

      \item In conclusion, while unsupervised methods might be useful in exploratory phases or when labeled data is not
            available, the specific nature of ttH event classification, combined with the availability of simulated
            labeled data and theoretical grounding, makes supervised learning a well-suited and likely more effective
            approach.
\end{enumerate}

Clustering, as an unsupervised learning technique, can be applied to the context of our task, but with some significant
caveats. Here's how it might work and the challenges involved:

\begin{enumerate}
      \item Application: Clustering could be used to group collision events based on similarities in their observable
            features, without prior knowledge of labels. By identifying patterns in the data, you might uncover groups
            that correspond to different types of events, such as ttH and others.

      \item Prediction: Once the clusters have been identified, you could theoretically assign labels to those clusters
            based on expert knowledge or additional analysis. These labels could then be used to classify new data.
            However, this is where the challenges and limitations become apparent.

      \item Challenges and Limitations: Lack of Ground Truth: Without labeled data to guide the clustering, there's no
            straightforward way to ensure that the clusters align with the actual underlying physics. The clusters might
            correspond to some other aspects of the data that aren't relevant to the classification task.

      \item Ambiguity: The boundaries between clusters may be ambiguous, leading to uncertainty in how to label the
            clusters. This could result in less accurate classifications.

      \item Model Complexity: Interpreting and validating the clusters might require significant expertise and additional
            modeling, making it a more complex and potentially less reliable approach than supervised learning.

      \item Predictive Power: Without a clear mapping from features to labels (as in supervised learning), the predictive
            power of a clustering-based model might be lower, especially if the clusters do not correspond well to the
            underlying physics.

      \item Not Directly Aligned with the Task: Clustering is generally used for exploratory data analysis and pattern
            recognition rather than direct classification. Adapting it to a classification task like identifying ttH
            events may require significant modification and may not be as efficient or effective as using a method
            specifically designed for classification, such as supervised learning.
\end{enumerate}

\clearpage \subsection{Features used}

\subsubsection{\texttt{lep\_E\_X}} The energy of the Xth lepton.

\subsubsection{\texttt{DRjj\_lead}} $\Delta R$ between the two leading jets. $\Delta R$ is a distance metric in the
$\eta-\phi$ space frequently used in particle physics.

\subsubsection{\texttt{Ptll01}} The transverse momentum of the dilepton system made up of the two leading leptons.

\subsubsection{\texttt{lep\_nTrackParticles\_X}} The number of track particles associated with the Xth lepton.

\subsubsection{\texttt{custTrigMatch\_LooseID\_FCLooseIso\_DLT}} Custom trigger matching for loosely identified and loosely
isolated leptons, likely related to the dilepton trigger

\subsubsection{\texttt{Mll01}} The invariant mass of the two leading leptons.

\subsubsection{\texttt{Mlll012}} The invariant mass of the three leading leptons.

\subsubsection{\texttt{total\_charge}} The sum of the electric charges of the particles in the event.

\subsubsection{\texttt{HT}} The scalar sum of the transverse momenta of all jets in the event.

\subsubsection{\texttt{HT\_lep}} The scalar sum of the transverse momenta of all leptons in the event.

\subsubsection{\texttt{lep\_Eta\_X}} The pseudorapidity of the Xth lepton.

\subsubsection{\texttt{nTaus\_OR\_Pt25}} The number of overlapping-removed taus with a transverse momentum above 25 GeV.

\subsubsection{\texttt{nFwdJets\_OR}} The number of overlapping-removed forward jets.

\subsubsection{\texttt{MLepMet}} The invariant mass of a lepton and the missing transverse energy vector.

\subsubsection{\texttt{taus\_DL1r\_X}} The DL1r score for the Xth tau.

\subsubsection{\texttt{lep\_isolationLoose\_VarRad\_X}} Indicates whether a lepton (where X refers to the lepton index)
passes an isolation cut with a variable radius. Looser isolation cuts allow more nearby activity in the detector.

\subsubsection{\texttt{lep\_EtaBE2\_X}} The pseudorapidity of the Xth lepton in the second layer of the electromagnetic
calorimeter.

\subsubsection{\texttt{HT\_fwdJets}} The scalar sum of the transverse momenta of all forward jets in the event.

\subsubsection{\texttt{taus\_width\_X}} The width of the Xth tau.

\subsubsection{\texttt{nJets\_OR\_DL1r\_85}} Count of jets that pass overlap removal (OR) and are b-tagged according to the
DL1r algorithm at the 85\% working point.

\subsubsection{\texttt{lep\_nInnerPix\_X}} Number of hits in the inner pixel detector associated with the lepton, where X
refers to the lepton index.

\subsubsection{\texttt{met\_phi}} The azimuthal angle of the missing transverse energy in the event.

\subsubsection{\texttt{DeltaR\_max\_lep\_bjet77}} The maximum DeltaR value between a lepton and a b-tagged jet. The "77"
may refer to the working point of the b-tagging algorithm.

\subsubsection{\texttt{MbX}} Invariant mass associated with the leading b-jet in the event

\subsubsection{\texttt{lep\_RadiusCO\_X}} Possibly the radius of the cone used for isolation of the lepton, or
alternatively a parameter associated with the trajectory of the lepton.

\subsubsection{\texttt{lep\_Mtrktrk\_atConvV\_CO\_X}} The invariant mass of track pairs at the conversion vertex for lepton
X. This might be related to photon conversions into an electron-positron pair.

\subsubsection{\texttt{lep\_Z0SinTheta\_X}} The z0 impact parameter times the sine of the lepton's polar angle.

\subsubsection{\texttt{lep\_Pt\_X}} The transverse momentum of the Xth lepton.

\subsubsection{\texttt{mjjMax\_frwdJet}} The maximum invariant mass of a pair of forward jets.

\subsubsection{\texttt{dilep\_type}} The type of dilepton event (e.g., $ee$, $\mu e$, $\mu \mu$).

\subsubsection{\texttt{eta\_frwdjet}} The pseudorapidity of the forward jet.

\subsubsection{\texttt{Mlb}} Invariant mass of a lepton and a b-jet.

\subsubsection{\texttt{taus\_RNNJetScoreSigTrans\_X}} Transformed RNN-based score for tau lepton, possibly to better
separate signal from background.

\subsubsection{\texttt{minDeltaR\_LJ\_X}} The minimum $\Delta R$ distance between the Xth lepton and any jet in the event.

\subsubsection{\texttt{nTaus\_OR}} Number of tau leptons that pass overlap removal. Overlap removal is a step in particle
reconstruction where, for instance, an object identified as both a jet and a tau would be considered only as one or the
other.

\subsubsection{\texttt{DeltaR\_min\_lep\_jet}} The minimum $\Delta R$ distance between a lepton and a jet in the event.

\subsubsection{\texttt{lep\_sigd0PV\_X}} Significance of the transverse impact parameter (d0) of the lepton X with respect
to the primary vertex (PV). This is a common variable for distinguishing prompt particles produced in the primary
collision from secondary particles produced in a decay.

\subsubsection{\texttt{taus\_eta\_X}} The pseudorapidity of the Xth tau.

\subsubsection{\texttt{HT\_jets}} The scalar sum of the transverse momenta of all jets (not forward jets) in the event.

\subsubsection{\texttt{lep\_Phi\_X}} The azimuthal angle (in radians) of the Xth lepton.

\subsubsection{\texttt{bTagSF\_weight\_DL1r\_85}} A weight applied to events based on the scale factor for b-tagging using
the DL1r algorithm at an 85\% efficiency working point. This scale factor corrects the b-tagging efficiency in Monte
Carlo simulations to match that observed in real data.

\subsubsection{\texttt{lep\_chargeIDBDTResult\_recalc\_rel207\_tight\_X}} The outcome of a BDT-based charge identification
for a lepton, recalculated with some specific settings, and applying a 'tight' threshold.

\subsubsection{\texttt{taus\_phi\_X}} The azimuthal angle (in radians) of the Xth tau.

\subsubsection{\texttt{taus\_passJVT\_X}} A boolean flag indicating whether the Xth tau passes the jet vertex tightness
(JVT) requirement.

\subsubsection{\texttt{jets\_eta}} The pseudorapidity of the jets (array).

\subsubsection{\texttt{taus\_charge\_X}} The charge of the Xth tau.

\subsubsection{\texttt{passPLIVTight\_X}} Boolean flag indicating if a lepton with high transverse momentum passes the
"tight" criteria of the Prompt Lepton Veto (PLIV), a tool for identifying non-prompt light leptons.

\subsubsection{\texttt{lep\_Mtrktrk\_atPV\_CO\_X}} The invariant mass of track pairs at the primary vertex for lepton X.
This could be related to certain types of particle decays happening at the primary collision vertex.

\subsubsection{\texttt{taus\_JetRNNSigMedium\_X}} RNN-based score for tau lepton, used to distinguish tau leptons from
jets, with 'medium' selection criteria.

\subsubsection{\texttt{minOSMll}} The minimum invariant mass of oppositely-signed dilepton pairs.

\subsubsection{\texttt{lep\_ID\_X}} The identification number for the Xth lepton.

\subsubsection{\texttt{Mllll0123}} The invariant mass of the four leading leptons.

\subsubsection{\texttt{custTrigSF\_TightElMediumMuID\_FCLooseIso\_DLT}} Custom trigger scale factor, for events with a
tight electron and a medium muon, both of which are loosely isolated, likely related to the dilepton trigger (DLT).

\subsubsection{\texttt{best\_Z\_Mll}} The invariant mass of the dilepton system that is closest to the Z boson mass.

\subsubsection{\texttt{met\_met}} The missing transverse energy in the event.

\subsubsection{\texttt{MtLep1Met}} Transverse mass between the leading lepton and missing transverse energy. Transverse
mass is often used in searches for particles that decay to a lepton and a neutrino.

\subsubsection{\texttt{lep\_ambiguityType\_X}} Type of ambiguity for lepton identification, where X refers to the lepton
index. Ambiguity could arise from several factors, such as a single track matching with multiple reconstructed
particles.

\subsubsection{\texttt{jets\_phi}} The azimuthal angle (in radians) of the jets (array).

\subsubsection{\texttt{lep\_isMedium\_X}} Boolean flag indicating if a lepton passes the 'medium' selection criteria.

\subsubsection{\texttt{taus\_RNNJetScore\_X}} RNN-based score for tau lepton, used to distinguish tau leptons from jets.

\subsubsection{\texttt{MtLepMet}} The transverse mass of a lepton and the missing transverse energy vector.

\subsubsection{\texttt{DeltaR\_min\_lep\_jet\_fwd}} The minimum $\Delta R$ distance between a lepton and a forward jet in the event.

\subsubsection{\texttt{jets\_e}} The energy of the jets (array).

\subsubsection{\texttt{minOSSFMll}} The minimum invariant mass of oppositely-signed, same-flavor dilepton pairs.

\subsubsection{\texttt{nJets\_OR}} The number of overlapping-removed jets.

\subsubsection{\texttt{total\_leptons}} The total number of leptons in the event.

\subsubsection{\texttt{taus\_numTrack\_X}} The number of tracks associated with the Xth tau.

\subsubsection{\texttt{HT\_taus}} Scalar sum of the transverse momenta ($P_t$) of all tau leptons in the event.

\subsubsection{\texttt{taus\_passEleOLR\_X}} A boolean flag indicating whether the Xth tau passes the electron overlap
removal.

\subsubsection{\texttt{HT\_inclFwdJets}} The scalar sum of the transverse momenta of all jets, including forward jets, in
the event.

\subsubsection{\texttt{DRll01}} The $\Delta R$ distance between the two leading leptons.

\subsubsection{\texttt{taus\_JetRNNSigLoose\_X}} RNN-based score for tau lepton, used to distinguish tau leptons from
jets, with 'loose' selection criteria.

\subsubsection{\texttt{taus\_pt\_X}} The transverse momentum of the Xth tau.

\subsubsection{\texttt{bTagSF\_weight\_DL1r\_77}} A weight applied to events based on the scale factor for b-tagging using
the DL1r algorithm at an 77\% efficiency working point. This scale factor corrects the b-tagging efficiency in Monte
Carlo simulations to match that observed in real data.

\subsubsection{\texttt{flag\_JetCleaning\_LooseBad}} A flag variable indicating whether a jet passes a loose cleaning cut
to remove bad or noisy jets from the analysis.

\subsubsection{\texttt{taus\_fromPV\_X}} A boolean flag indicating whether the Xth tau comes from the primary vertex.

\subsubsection{\texttt{best\_Z\_other\_MtLepMet}} The transverse mass between the lepton and missing transverse energy for
the event that best reconstructs a Z boson using other criteria.

\subsubsection{\texttt{nJets\_OR\_DL1r\_77}} Count of jets that pass overlap removal (OR) and are b-tagged according to the
DL1r algorithm at the 77\% working point.

\subsubsection{\texttt{jets\_pt}} The transverse momentum of the jets (array).

\subsubsection{\texttt{lep\_isTightLH\_X}} Boolean flag indicating if a lepton passes the 'tight' Likelihood-based
identification criteria.

\subsubsection{\texttt{taus\_JetRNNSigTight\_X}} RNN-based score for tau lepton, used to distinguish tau leptons from
jets, with 'tight' selection criteria.

\subsubsection{\texttt{sumPsbtag}} The sum of b-tagging weights for jets in the event.

\subsubsection{\texttt{taus\_decayMode\_X}} The decay mode of the Xth tau.

\subsubsection{\texttt{dEta\_maxMjj\_frwdjet}} The maximum difference in pseudorapidity ($\eta$) between two forward jets.

\subsubsection{\texttt{max\_eta}} The maximum pseudorapidity among all particles in the event.

\subsubsection{\texttt{best\_Z\_other\_Mll}} The invariant mass of the dilepton system that is closest to the Z boson mass,
not considering the leading leptons.

\subsubsection{\texttt{taus\_passEleBDT\_X}} Flag indicating if a tau lepton passes the Electron Boosted Decision Tree
discriminator.

\begin{figure}[hbtp]
    \centering
    \includegraphics[width=\textwidth]{figures/ml/features/top20.pdf}
    \caption{Feature importance for the top 20 most important features. Feature importance was calculated using the
        \gls{ig} method \cite{ig}.}
    \label{fig:feature_importance}
\end{figure}

\clearpage
\section{Training optimizations}

Here we list some additional details that do not influence the quality of the optimization itself, but have a rather
tangent relation to it.

As we have used large \glspl{nn} and a large training set, the training times are quite long. Without some necessary
optimization, experiments take extremely long time to complete. We have used the following techniques to speed up the
training:

\subsection{Mixed precision training}

Mixed precision training is a technique in deep learning that leverages the benefits of both low-precision and
high-precision numerical representations to accelerate model training and improve overall efficiency. It involves using
a combination of reduced-precision (such as 16-bit) and full-precision (such as 32-bit) floating-point computations
during the training process. By employing reduced precision for certain computations, such as matrix multiplications,
mixed precision training can significantly speed up the training process while maintaining a comparable level of
accuracy. This approach is especially useful when training large-scale models with massive amounts of data, as it
reduces memory usage, allows for faster computations, and enables the use of larger batch sizes. Overall, mixed
precision training is a valuable technique that helps us achieve faster and more efficient deep learning models, leading
to quicker iteration cycles and advancements in various fields, including computer vision, natural language processing,
and reinforcement learning.

% We have observed an about 2.5x speedup when using mixed precision training, which can be
% seen from the \autoref{tab:additional_optimization}.

\subsection{PyTorch 2.0 and torch.compile()}

\verb|torch.compile()| is a feature introduced in PyTorch 2.0 \cite{pytorch} that aims to improve the performance of
PyTorch code by JIT-compiling it into optimized kernels. It allows PyTorch code to run faster while requiring minimal
code changes. The \verb|torch.compile()| supports arbitrary PyTorch code, control flow, and mutation, and comes with
experimental support for dynamic shapes. By using \verb|torch.compile()|, developers can optimize their PyTorch code
without sacrificing flexibility or ease of use.  This feature is particularly useful for boosting the performance of
PyTorch models during training and inference.

% We have observed an about 2.5x speedup when using \verb|torch.compile()|,
% which can be seen from the \autoref{tab:additional_optimization}.

\subsection{FlashAttention}

FlashAttention is a fast and memory-efficient exact attention algorithm that aims to improve the training speed and
quality of models with long sequences in machine learning applications. It incorporates IO-awareness, which involves
dividing operations between faster and slower levels of GPU memory to optimize performance. By reordering the attention
computation and leveraging classical techniques such as tiling and recomputation, FlashAttention significantly speeds up
the attention process and reduces memory usage from quadratic to linear in sequence length. This algorithm outperforms
other exact attention algorithms in terms of training speed and model quality, especially when dealing with long
sequences. It achieves faster end-to-end training time and higher quality models by accounting for GPU memory reads and
writes, resulting in improved performance and reduced compute complexity. FlashAttention is a valuable tool for
researchers and practitioners working with attention mechanisms in machine learning, enabling them to train models more
efficiently and effectively.

% We have observed an about 2.5x speedup when using FlashAttention, which can be seen from
% the \autoref{tab:additional_optimization}.

% To illustrate the benefits of all the optimizations, we have trained several neural networks of different sizes with
% different optimization included or excluded.  Results are summarized on the \autoref{tab:additional_optimization}. When
% all the optimizations are included, for the largest network we used, we observe a roughtly 6x speedup.

%
% We actually didn't measure exactly but it's a huge factor :)
%

% \begin{table}[htbp]
%     \centering
%     \begin{tabular}{ccccc}
%         \toprule
%         Network Size & Mixed Precision & \verb|torch.compile()| & FlashAttention & Speedup \\
%         \midrule
%         Small        &                 &                        &                & 1.0     \\
%         Small        & +               &                        &                & 2.0     \\
%         Small        &                 & +                      &                & 2.5     \\
%         Small        &                 &                        & +              & 4.0     \\
%         Small        & +               & +                      &                & 2.5     \\
%         Small        & +               &                        & +              & 5.0     \\
%         Small        &                 & +                      & +              & 5.0     \\
%         Small        & +               & +                      & +              & 6.0     \\
%         \midrule
%         Medium       &                 &                        &                & 1.0     \\
%         Medium       & +               &                        &                & 2.0     \\
%         Medium       &                 & +                      &                & 2.5     \\
%         Medium       &                 &                        & +              & 4.0     \\
%         Medium       & +               & +                      &                & 2.5     \\
%         Medium       & +               &                        & +              & 5.0     \\
%         Medium       &                 & +                      & +              & 5.0     \\
%         Medium       & +               & +                      & +              & 6.0     \\
%         \midrule
%         Large        &                 &                        &                & 1.0     \\
%         Large        & +               &                        &                & 2.0     \\
%         Large        &                 & +                      &                & 2.5     \\
%         Large        &                 &                        & +              & 4.0     \\
%         Large        & +               & +                      &                & 2.5     \\
%         Large        & +               &                        & +              & 5.0     \\
%         Large        &                 & +                      & +              & 5.0     \\
%         Large        & +               & +                      & +              & 6.0     \\
%         \bottomrule
%     \end{tabular}
%     \caption{Additional optimization results for different neural networks. Small network has 3 blocks, 1.5 million
%         parameters in total, medium network has 6 blocks, 3 million parameters in total, and large network has 9 blocks,
%         4.5 million parameters in total.}
%     \label{tab:additional_optimization}
% \end{table}

\section{Training speed}
\label{appendix:training-speed}

Training large transformers takes long and has a large carbon footprint.

\section{Training times}

Training transformers takes long and has a large carbon footprint.