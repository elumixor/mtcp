\section{Training optimizations}

Here we list some additional details that do not influence the quality of the optimization itself, but have a rather
tangent relation to it.

As we have used large \glspl{nn} and a large training set, the training times are quite long. Without some necessary
optimization, experiments take extremely long time to complete. We have used the following techniques to speed up the
training:

\subsection{Mixed precision training}

Mixed precision training is a technique in deep learning that leverages the benefits of both low-precision and
high-precision numerical representations to accelerate model training and improve overall efficiency. It involves using
a combination of reduced-precision (such as 16-bit) and full-precision (such as 32-bit) floating-point computations
during the training process. By employing reduced precision for certain computations, such as matrix multiplications,
mixed precision training can significantly speed up the training process while maintaining a comparable level of
accuracy. This approach is especially useful when training large-scale models with massive amounts of data, as it
reduces memory usage, allows for faster computations, and enables the use of larger batch sizes. Overall, mixed
precision training is a valuable technique that helps us achieve faster and more efficient deep learning models, leading
to quicker iteration cycles and advancements in various fields, including computer vision, natural language processing,
and reinforcement learning.

% We have observed an about 2.5x speedup when using mixed precision training, which can be
% seen from the \autoref{tab:additional_optimization}.

\subsection{PyTorch 2.0 and torch.compile()}

\verb|torch.compile()| is a feature introduced in PyTorch 2.0 \cite{pytorch} that aims to improve the performance of
PyTorch code by JIT-compiling it into optimized kernels. It allows PyTorch code to run faster while requiring minimal
code changes. The \verb|torch.compile()| supports arbitrary PyTorch code, control flow, and mutation, and comes with
experimental support for dynamic shapes. By using \verb|torch.compile()|, developers can optimize their PyTorch code
without sacrificing flexibility or ease of use.  This feature is particularly useful for boosting the performance of
PyTorch models during training and inference.

% We have observed an about 2.5x speedup when using \verb|torch.compile()|,
% which can be seen from the \autoref{tab:additional_optimization}.

\subsection{FlashAttention}

FlashAttention is a fast and memory-efficient exact attention algorithm that aims to improve the training speed and
quality of models with long sequences in machine learning applications. It incorporates IO-awareness, which involves
dividing operations between faster and slower levels of GPU memory to optimize performance. By reordering the attention
computation and leveraging classical techniques such as tiling and recomputation, FlashAttention significantly speeds up
the attention process and reduces memory usage from quadratic to linear in sequence length. This algorithm outperforms
other exact attention algorithms in terms of training speed and model quality, especially when dealing with long
sequences. It achieves faster end-to-end training time and higher quality models by accounting for GPU memory reads and
writes, resulting in improved performance and reduced compute complexity. FlashAttention is a valuable tool for
researchers and practitioners working with attention mechanisms in machine learning, enabling them to train models more
efficiently and effectively.

% We have observed an about 2.5x speedup when using FlashAttention, which can be seen from
% the \autoref{tab:additional_optimization}.

% To illustrate the benefits of all the optimizations, we have trained several neural networks of different sizes with
% different optimization included or excluded.  Results are summarized on the \autoref{tab:additional_optimization}. When
% all the optimizations are included, for the largest network we used, we observe a roughtly 6x speedup.

%
% We actually didn't measure exactly but it's a huge factor :)
%

% \begin{table}[htbp]
%     \centering
%     \begin{tabular}{ccccc}
%         \toprule
%         Network Size & Mixed Precision & \verb|torch.compile()| & FlashAttention & Speedup \\
%         \midrule
%         Small        &                 &                        &                & 1.0     \\
%         Small        & +               &                        &                & 2.0     \\
%         Small        &                 & +                      &                & 2.5     \\
%         Small        &                 &                        & +              & 4.0     \\
%         Small        & +               & +                      &                & 2.5     \\
%         Small        & +               &                        & +              & 5.0     \\
%         Small        &                 & +                      & +              & 5.0     \\
%         Small        & +               & +                      & +              & 6.0     \\
%         \midrule
%         Medium       &                 &                        &                & 1.0     \\
%         Medium       & +               &                        &                & 2.0     \\
%         Medium       &                 & +                      &                & 2.5     \\
%         Medium       &                 &                        & +              & 4.0     \\
%         Medium       & +               & +                      &                & 2.5     \\
%         Medium       & +               &                        & +              & 5.0     \\
%         Medium       &                 & +                      & +              & 5.0     \\
%         Medium       & +               & +                      & +              & 6.0     \\
%         \midrule
%         Large        &                 &                        &                & 1.0     \\
%         Large        & +               &                        &                & 2.0     \\
%         Large        &                 & +                      &                & 2.5     \\
%         Large        &                 &                        & +              & 4.0     \\
%         Large        & +               & +                      &                & 2.5     \\
%         Large        & +               &                        & +              & 5.0     \\
%         Large        &                 & +                      & +              & 5.0     \\
%         Large        & +               & +                      & +              & 6.0     \\
%         \bottomrule
%     \end{tabular}
%     \caption{Additional optimization results for different neural networks. Small network has 3 blocks, 1.5 million
%         parameters in total, medium network has 6 blocks, 3 million parameters in total, and large network has 9 blocks,
%         4.5 million parameters in total.}
%     \label{tab:additional_optimization}
% \end{table}
