\clearpage

\section{Event Weighting}
\label{appendix:weights}

\begin{align}
    \text{Channels} = \{
     & 364250, 364253, 364254, 364255, 364283, 364284, 364285,  \nonumber \\
     & 364286, 364287, 363355, 363356, 363357, 363358, 363359,            \\
     & 363360, 363489, 345705, 345706, 345715, 345718, 345723\} \nonumber
\end{align}

\begin{equation}
    \text{Weight}_{\text{nJets}} = \begin{cases}
        1.0,      & \text{if } \texttt{nJets\_OR} = 0    \\
        0.986980, & \text{if } \texttt{nJets\_OR} = 1    \\
        0.853062, & \text{if } \texttt{nJets\_OR} = 2    \\
        0.785437, & \text{if } \texttt{nJets\_OR} = 3    \\
        0.741692, & \text{if } \texttt{nJets\_OR} = 4    \\
        0.709992, & \text{if } \texttt{nJets\_OR} = 5    \\
        0.685452, & \text{if } \texttt{nJets\_OR} = 6    \\
        0.665613, & \text{if } \texttt{nJets\_OR} \geq 7
    \end{cases}
\end{equation}

\begin{equation}
    \texttt{XXX\_VV\_NJET} = \begin{cases}
        \text{Weight}_{\text{nJets}}, & \text{if } \texttt{mcChannelNumber} \in \text{Channels} \\
        1.0,                          & \text{otherwise}
    \end{cases}
\end{equation}

\begin{equation}
    \text{YearLuminosity} = \begin{cases}
        36646.74, & \text{if } \texttt{RunYear} = 2015 \text{ or } \texttt{RunYear} = 2016 \\
        44630.6,  & \text{if } \texttt{RunYear} = 2017                                     \\
        58791.6,  & \text{if } \texttt{RunYear} = 2018
    \end{cases}
\end{equation}

\begin{align}
    \text{Event weight} = w =
     & \quad \texttt{YearLuminosity}                                       \nonumber \\
     & \times \frac{\sigma}{\mathcal{L}}                                   \nonumber \\
     & \times \texttt{custTrigSF\_LooseID\_FCLooseIso\_SLTorDLT}           \nonumber \\
     & \times \texttt{weight\_pileup}                                      \nonumber \\
     & \times \texttt{jvtSF\_customOR}                                     \nonumber \\
     & \times \texttt{bTagSF\_weight\_DL1r\_85}                            \nonumber \\
     & \times \texttt{XXX\_VV\_NJET}                                       \nonumber \\
     & \times \texttt{weight\_mc}                                          \nonumber \\
     & \times \texttt{lep\_SF\_CombinedTight\_0}                           \nonumber \\
     & \times \texttt{lep\_SF\_CombinedTight\_1}                           \nonumber \\
     & \times \texttt{lepSF\_PLIV\_Prompt\_0}                              \nonumber \\
     & \times \texttt{lepSF\_PLIV\_Prompt\_1}                              \nonumber \\
     & \times \frac{1}{\sum_{w \sim \text{\acrshort{mc}}} w}\,,
\end{align}

where $\mathcal{L} = \text{Luminosity} = 140068.94$, $\sigma$ is the cross-section ($\texttt{xs}$), and $\sum_{w \sim
        \text{\acrshort{mc}}} w$ is the sum of the weights of all simulated events ($\texttt{totalEventsWeighted}$).



% Luminosity: The total number of simulated events often does not correspond to the luminosity of the real data. Events
% are therefore weighted to correspond to the correct integrated luminosity. If $L_{data}$ is the integrated luminosity of
% the data and $L_{\gls{mc}}$ is the integrated luminosity of the \gls{mc} sample, then the luminosity weight for an event is $w_{L} =
%     L_{data}/L_{\gls{mc}}$.

% Cross-Section: Different processes have different probabilities (cross-sections) of occurring. The ratio of the
% cross-sections in data and simulation $\sigma_{data}/\sigma_{\gls{mc}}$ is used as a weight.

% Detector Effects: The detector response is not always perfectly simulated. Therefore, weights are applied to correct for
% known discrepancies in detector efficiencies and energy resolutions between the data and the simulation.

% Pileup: Pileup refers to additional proton-proton collisions that occur simultaneously with the event of interest.
% Pileup can significantly affect the event reconstruction. Pileup weights are used to match the pileup distribution in
% the data.

% Higher-order corrections: Theoretical predictions often include higher-order corrections, known as K-factors, to account
% for processes beyond the leading-order approximation used in the simulation.