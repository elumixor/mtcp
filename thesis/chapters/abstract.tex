\chapter*{Abstract}
\addcontentsline{toc}{chapter}{Abstract}

This research is centered on the goal of accurately selecting events produced from the interaction of two gluons in a
proton-proton collision at the \acrshort{lhc}, resulting in a top-antitop pair and a Higgs boson, a process known as \tth.

With data gathered from the \acrshort{lhc}'s \acrshort{atlas} detector, the study aims to distinguish these \tth events
from those generated by other processes. To this end, we employ a deep learning approach, specifically a \acrshort{ftt}
architecture, for the event selection. The use of such a machine learning method enhances our ability to identify \tth
events accurately, leading to an improved signal-to-noise ratio and statistical significance, thereby contributing to
our understanding of the Higgs boson's properties. Compared to the previous results presented by \cite{severin} and
\cite{jan}, our approach achieves a higher statistical significance of \placeholder{XXX}, marking a notable improvement.

An integral part of this research is the evaluation of both statistical and systematic uncertainties associated with
this event selection process. The findings and methodologies presented in this thesis offer promising advancements in
particle physics event selection, contributing to the ATLAS collaboration's ongoing endeavors to probe the fundamental
properties of the universe.

\vspace{3mm}
\noindent
\textbf{Keywords:}
CERN, ATLAS, Higgs boson, Machine Learning, Neural Networks