\section{Problem Formulation}
\label{sec:formulation}

\todo{proofread}

\glsreset{erm}
\subsection[Empirical Risk Minimization]{\gls{erm}}

As stated before, our primary objective in this thesis is to distinguish the \tth events from other events of the other
detected by the \gls{atlas} detector. Given an observation $\bm{x} \in \mathcal{X}$, we want to predict its
corresponding class label $y \in \mathcal{Y}$. Here $\mathcal{X}$ denotes the space of all possible observations (in
our case this corresponds to the different measurements of the event), and $\mathcal{Y}$ denotes the space of all
the class labels we are differentiating between. We can further split the problem into either a binary classification
(seeking to differentiate between \tth (signal) and not \tth (background)) or a multi-class classification (seeking to
correctly discriminate between each of the processes - \tth, \ttw, \ttz, \ttbar, etc.).

As we approach this task as a supervised learning problem (recall \autoref{sec:mc}), we assume that a set of labeled
observations $\ttrn = {(\bm{x}_i, y_i)}_{i=1}^N$ is provided.  Here $\bm{x}_i \in \mathbf{X}$ is a feature vector
representing different properties (features) of an event and $y_i \in \mathbf{Y}$ is its corresponding true class label.
We assume there exists a joint probability distribution $P(\bm{x}, y)$ over the observations $\bm{x}$ and their
corresponding class labels $y$. Then, we require the examples in the training set $(\bm{x}_i, y_i) \in \ttrn$ to be
drawn \gls{iid} from the joint distribution $P(\bm{x}, y)$.


We also assume that there is a non-negative real-valued loss function $L(y, \hat{y})$ that quantifies the
discrepancy between the true label $y$ and the predicted label $\hat{y}$. The common example of such a loss function
would be a zero-one loss function, which is defined as

\begin{equation}
    L_{0/1}(y, \hat{y}) = \begin{cases}
        0 & \text{if } y = \hat{y} \\
        1 & \text{otherwise}
    \end{cases}\,.
\end{equation}

The goal is to find the best hypothesis $h^* \in \mathcal{H}:
    \mathbf{X} \rightarrow \mathbf{Y}$ that would minimize the expected loss over the joint distribution $P(\bm{x}, y)$:

\begin{equation}
    h^* = \argmin_{h \in \mathcal{H}} \mathbb{E}_{(\bm{x}, y) \sim P(\bm{x}, y)}[L(y, h(\bm{x}))]\,.
\end{equation}

In practice, we do not have access to the joint distribution $P(\bm{x}, y)$, but only to the training set $\ttrn$. To
tackle this problem, we use the \gls{erm} principle \cite{risk-minimization}, which states that the best hypothesis
$h^*$ is the one that minimizes the empirical risk over the training set $\ttrn$:

\begin{equation}
    \hat{h} = \argmin_{h \in \mathcal{H}} R_\ttrn(h) = \argmin_{h \in \mathcal{H}} \frac{1}{N} \sum_{i=1}^{N} L(y_i, h(\bm{x}_i))\,.
\end{equation}










\subsection{Validation and Test Sets}

Consider, for example the following "cheating" classifier:

\begin{equation}
    h(\bm{x}) = \begin{cases}
        y_i & \text{if } \exists i : \bm{x} = \bm{x}_i \\
        y_0 & \text{otherwise}
    \end{cases}\,.
\end{equation}


This classifier would have zero empirical risk, but would perform poorly on unseen data (would not generalize well).
This is referred to as overfitting (see \autoref{fig:standard-losses}). Generally, in case of an unconstrained
hypothesis space $\mathcal{H}$, we have no guarantee that the empirical risk $R_\ttrn(h)$ is a good approximation of the
true risk $R(h)$.


Specifically, the problem in this case is that the prediction $\hat{y}_i$ depends not only on the observation
$\bm{x}_i$, but also on the labels $y_1, \dots, y_N$. Consider, for example, a \gls{nn} classifier $h_\nnparams$ with trainable
parameters $\nnparams$. When training $h_\nnparams$ on the training set by back-propagation, $\nnparams$  becomes implicitly
conditioned on the true labels $y^1, \dots, y^s$ that the network has encountered before ($s$ denotes the training step).
This violates the \gls{iid} assumption and thus the empirical risk $R_\ttrn(h_\nnparams)$ is not a good approximation of
the true risk $R(h_\nnparams)$ anymore.


To address this issue and more accurately assess the generalization ability of the classifier $h$, We would need a
separate set $\tval \sim P(\bm{x}, y)$ that would provide an unbiased estimate of the true risk $R(h)$. This set is
called the validation set. Validation set is used to compare the performance of different
classifiers. Consider two classifiers $h_1$ and $h_2$, where the risk on the training set is $R_\ttrn(h_1) <
    R_\ttrn(h_2)$, but the risk on the validation set is $R_\tval(h_1) > R_\tval(h_2)$. In this case, we would prefer the
classifier $h_2$ over $h_1$ as it generalizes better to unseen data. The specific case is often seen with the \gls{nn}
classifiers, where the classifier $h$ is parametrized by $\nnparams$. Then essentially we compare $h_1 = h_{\nnparams_1}$ and
$h_2 = h_{\nnparams_2}$, where $\nnparams_1$ and $\nnparams_2$ are two different sets of parameters. The process of selecting
the best classifier $h^*$ from a set of classifiers $\mathcal{H}$ is called model selection.

\begin{figure}[htb]
    \centering
    \begin{minipage}[t][\height][t]{0.47\textwidth}
        \includegraphics[width=\textwidth]{figures/ml/training/standard-ts-losses.png}
        \caption{\acrshort{ftt} with 2 blocks and 256 embedding size training on the standard training set.}
        \label{fig:standard-losses}
    \end{minipage}
    \hfill
    \begin{minipage}[t][\height][t]{0.47\textwidth}
        \includegraphics[width=\textwidth]{figures/ml/training/extended-ts-losses.png}
        \caption{\acrshort{ftt} with 5 blocks, 256 embedding size, and 20\% dropout training on the \emph{extended} training
            set.} \label{fig:extended-losses}
    \end{minipage}
    \caption[Training and validation losses during the training process with and without extended training set.]
    {Training and validation losses during the training process with and without extended training set.
        \autoref{fig:standard-losses} shows the training of the \acrshort{ftt} with 2 blocks (see \autoref{sec:ftt}) on the
        standard training set. Because of the lack of training samples, and high capacity of the model, we observe
        overfitting. The training loss continues to decrease while the validation loss starts to increase. The
        checkpoint with the validation loss is the lowest is then used as the final model. This is referred to as early
        stopping. The best way to prevent overfitting is to get more training data (see
        \autoref{sec:extended-set}). If that is not possible, one might also consider augmentation techniques, or
        regularization - dropout (see \autoref{sec:dropout}), weight decay etc. \autoref{fig:extended-losses} shows the
        training of the \acrshort{ftt} with 5 blocks (see \autoref{sec:ftt}) on the extended training set. Also, the 20\%
        dropout is introduced. We observe now only a slight overfitting, which means that the model has generalized
        a lot better.}
    \label{fig:losses}
\end{figure}

The caveat of using the validation for model selection is that in doing so we are implicitly fitting to the validation set,
as now our best classifier $h^*$ is also conditioned on the evaluations of the other classifiers on the validation set.
To address the similar issue, a third set is normally introduced, called the test set $\ttst \sim P(\bm{x}, y)$. The test
set should be used only once to assess the performance of the fully-trained classifier $h^*$.




\todo{But then why we don't use the test set in our thesis???}





\subsection{Training}

The process of finding the best hypothesis $h^*$ is called training (or learning). In the context of \gls{erm}, this
further reduces to minimization of the empirical risk $R_\ttrn(h)$. As described before, the empirical risk is an
expectation of the loss function $L(y, h(\bm{x}))$ over the training set $\ttrn$. Thus, the choice of the loss function
will determine the available training algorithms.

This work focuses on training the \gls{nn} classifiers. \gls{nn} can be formally described as a parametric model
parametrized by a vector of parameters $\nnparams$. \glspl{nn} are generally composed of multiple layers
$f_{\nnparams_i}^1, \dots f_{\nnparams_L}^D$ ($D$ denotes the total number of layers - depth of the network), where
each layer $f_{\nnparams_i}^i$ is a parametric function parametrized by $\nnparams_i$.  \glspl{nn} can take different
architectures, a common one\footnote{In this thesis we use an adaptation of \glspl{resnet} to tabular data and
    \glspl{ftt}, which are both a special cases of feed-forward \glspl{nn}.} is a feed-forward \gls{nn} where the output
of the layer $f_{\nnparams_i}^i$ is fed as an input to the next layer $f_{\nnparams_{i+1}}^{i+1}$. The output
of the last layer $f_{\nnparams_L}^D$ is the output of the network $h_\nnparams$. Alternative to feed-forward
\glspl{nn} would be the network that contain cycles (e.g.  \glspl{rnn}\footnote{Overview of the different types of
    \glspl{rnn} \url{https://paperswithcode.com/methods/category/recurrent-neural-networks}.} \cite{rnn,lstm}).

The parameter vector of the whole neural network can be seen as a concatenation of the parameters of the individual
layers:

\begin{equation}
    \nnparams = \begin{bmatrix}
        \nnparams_1 \\
        \vdots      \\
        \nnparams_L
    \end{bmatrix}\,.
\end{equation}

The goal is then to find the optimal set of parameters $\nnparams^*$ that minimizes the empirical risk $R_\ttrn(h_\nnparams)$.

Training \glspl{nn} efficiently involves the use of gradient-based optimization algorithms where the gradient of the
empirical risk $R_\ttrn(h_\nnparams)$ with respect to the parameters $\nnparams$ is computed and used to update the parameters
$\nnparams$ in the direction of the steepest descent. The gradient of the empirical risk $R_\ttrn(h_\nnparams)$ with respect to
the parameters $\nnparams$ can be computed using the chain rule:

\begin{equation}
    \nabla_\nnparams R_\ttrn(h_\nnparams) = \nabla_\nnparams \frac{1}{N} \sum_{i=1}^{N} L(y_i, h_\nnparams(\bm{x}_i)) = \frac{1}{N} \sum_{i=1}^{N} \nabla_\nnparams L(y_i, h_\nnparams(\bm{x}_i))\,,
\end{equation}

and is essentially an average of the gradients of the loss function $L(y_i, h_\nnparams(\bm{x}_i))$ with respect to the
parameters $\nnparams$ over the training set $\ttrn$. During the update step, the parameters $\nnparams$ are updated in the
direction of the steepest descent:

\begin{equation}
    \nnparams \leftarrow \nnparams - \alpha \nabla_\nnparams R_\ttrn(h_\nnparams)\,,
\end{equation}

where $\alpha$ is the learning rate, a hyperparameter controlling the size of the update step. In practice,
instead of computing the gradient over the whole training set $\ttrn$, the gradient is computed on the so-called
mini-batches of the training data. This gradient, computed on the mini-batch acts as an unbiased estimate of the true
gradient. This approach is called \gls{sgd} and is the most common optimization algorithm used for training
\glspl{nn}\footnote{Aside from having low computational and
    memory requirements, being able to learn online, \gls{sgd} has some other advantages, such as being able to escape
    local minima, generalize better and provide the regularization effect - all consequences of an inherent noise in the
    gradient estimate.}.
Throughout our experiments we use an improvement of \gls{sgd} called \gls{adamw} \cite{adam, adamw} which is an adaptive
learning rate optimization algorithm that uses the first and second moments of the gradient to adapt the learning rate
dynamically.

Computation of the gradient of the loss function $L(y_i, h_\nnparams(\vec{x}_i))$ with respect to the parameters \nnparams
is done using the chain rule. The chain rule is a formula for computing the derivative of the composition of two or more
functions:

\begin{equation}
    \dv{(\vec{f} \circ \vec{g})}{\vec{x}} = \dv{\vec{f}}{\vec{g}} \dv{\vec{g}}{\vec{x}}\,,
\end{equation}

where $\vec{f}$ and $\vec{g}$ are functions of $\vec{x}$.

In the context of \glspl{nn}, the chain rule is used to compute the gradient of the loss function with respect to the
parameters \nnparams by an iterative approach. First, let the outputs of the individual layers are recorded during the
forward pass (or forward propagation):

\begin{align}
    \vec{z}^i & = \vec{f}^i(\vec{z}^{i-1}) \quad i = 1, \dots, L \\
    \vec{z}^0 & = \vec{x}\,,
\end{align}

where $\vec{z}^i$ is the output of the $i$-th layer and $\vec{z}^0 = \vec{x}$ is the input to the first layer. Then, we
can compute the gradients with respect to the outputs of the layers:

\begin{align}
    \vec{\delta}^D     & = \dv{L}{\vec{f}^D}                                                                              \\
    \vec{\delta}^{D-1} & = \dv{L}{\vec{f}^D} \dv{\vec{f}^D}{\vec{z}^{D-1}} = \vec{\delta}^D \dv{\vec{f}^D}{\vec{f}^{D-1}} \\
    \vec{\delta}^{D-2} & = \vec{\delta}^{D-1} \dv{\vec{f}^{D-1}}{\vec{f}^{D-2}}                                           \\
                       & \vdots \nonumber                                                                                 \\
    \vec{\delta}^1     & = \vec{\delta}^2 \dv{\vec{f}^2}{\vec{f}^1}\,.
\end{align}

Then, the
gradient of the loss function with respect to the parameters of each layer $\nnparams_i$ is computed as

\begin{align}
    \dv{L}{\nnparams_D}     & = \dv{L}{\vec{f}^D} \dv{\vec{f}^D}{\nnparams_{D}} = \vec{\delta}^D \dv{\vec{f}^D}{\nnparams_{D}} \\
    \dv{L}{\nnparams_{L-1}} & = \dv{L}{\vec{f}^D} \dv{\vec{f}^D}{\vec{f}^{D-1}} \dv{\vec{f}^{D-1}}{\nnparams_{L-1}} =\
    \vec{\delta}^{D-1} \dv{\vec{f}^{D-1}}{\nnparams_{D-1}}                                                                     \\
    \dv{L}{\nnparams_{D-2}} & = \vec{\delta}^{D-2} \dv{\vec{f}^{D-2}}{\nnparams_{D-2}}                                         \\
                            & \vdots \nonumber                                                                                 \\
    \dv{L}{\nnparams_{1}}   & = \vec{\delta}^1 \dv{\vec{f}^1}{\nnparams_{1}}\,.
\end{align}

The chain rule is the reason why \glspl{nn} are so successful in practice. It allows for the efficient computation of
the gradient even for very deep \glspl{nn}.






\subsection{Cross-entropy loss}

In order for the back-propagation to work, all the functions must be differentiable. The zero-one loss function that we
have used in the previous section does not conform to this requirement. In practice, when training \glspl{nn} on the
classification tasks, the cross-entropy loss function is used. Cross-entropy loss operates on the probabilities, rather
than on the predicted label, thus making it differentiable and suitable to be used in the back-propagation algorithm.
The cross-entropy loss function is defined as:

\begin{equation}
    L(y_i, \vec{h}(\vec{x}_i)) = -\sum_{j = 1}^{|\textbf{Y}|} \llbracket y_i = y_j \rrbracket \log(h_j(\vec{x}))\,.
\end{equation}

In certain scenarios, such as imbalanced datasets, it may be beneficial to apply different weights to different classes.
Class weights are used to give more importance to under-represented classes, effectively balancing the contribution of
each class to the overall loss. This helps the learning algorithm focus more on the minority class, which may be of
particular interest or significance.

For example, consider a medical diagnosis application where 95\% of the samples are negative ($y = 0$) and only 5\% are
positive ($y = 1$) for a specific condition. Training a model on this dataset without any adjustments may lead to a
classifier that almost always predicts the negative class, since it's encountering it much more often. Such a skewed
prediction can be problematic in critical applications, as missing the rare positive cases could have serious
consequences. To alleviate this issue, class weights can be introduced to the loss function to give equal importance to
both classes. The modified loss function is:

\begin{equation}
    L(y_i, \vec{h}(\vec{x}_i)) = -\sum_{j = 1}^{|\textbf{Y}|} \llbracket y_i = y_j \rrbracket w_j \log(h_j(\vec{x}))\,.
\end{equation}

Here weights $w_1, \dots, w_{|\textbf{Y}|}$ are assigned to each class, where $w_i$ is calculated as:

\begin{equation}
    w_i = \frac{|\{y \sim \textbf{Y} \mid y = i\}|}{|\textbf{Y}| \sum_{i = 1}^{|\textbf{Y}|} w_i} \quad i = 1, \dots, |\textbf{Y}|\,.
\end{equation}

Essentially, we count the number of examples of class $i$ and divide it by the total number of examples in the dataset.
Then we normalize\footnote{This is optional, but helpful - inverse frequencies can have a large range, especially if
    there is extreme imbalance between the classes. Keeping the weights in the [0, 1] range also help interpretability.}
the weights so that they sum up to 1.


Similarly, the way the cross-entropy is defined, we can also introduce a more refined sample-wise weighting. In this
case, the loss function would is defined as:

\begin{equation}
    L(y_i, \vec{h}(\vec{x}_i)) = -\sum_{j = 1}^{|\textbf{Y}|} \llbracket y_i = y_j \rrbracket w_i \log(h_j(\vec{x}))\,.
\end{equation}

Here we should note that $w_i$ is not the same as the class weight $w_i$ from the previous example. In this case, $w_i$
is a weight assigned to each sample, rather than to each class, which we note by using the same index $i$ as in the
$y_i$ and $\vec{x}_i$. This formulation is not commonly used, bt was explored by the previous analysis
\cite{severin,jan}, thus we include it here for completeness. More details are given in \autoref{sec:weights}.

