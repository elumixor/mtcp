\subsection{Feature Tokenizer + Transformer}
\label{sec:ftt}

\glsreset{ftt}
We adopt the \gls{ftt} as our primary architecture. The architecture was proposed in \cite{ft-transformer} and is an
adaptation of the famous transformer architecture \cite{transformer} to the tabular data.

\subsubsection{Transformer Architecture}

The transformer architecture was originally proposed for the \gls{nlp} processing tasks, but has since been applied
to almost every domain of machine learning. The transformer architecture is a fully-attentional architecture, which
means that it does not use any convolutional \cite{convolutional} or recurrent \cite{recurrent} layers. Instead, it uses
the attention\footnote{Here we use self-attention, which means that the keys and values are produced from the same
    source as queries.}
\footnote{We use the \emph{scaled} attention, which refers to a division of the scores by $\sqrt{d_k}$ before applying
    the $\text{softmax}$.}
mechanism \cite{attention} to learn the dependencies between the input features.



% \todo{1. we need to add the diagram here}




The crux of the transformer's power lies in its scaled dot-product attention mechanism, which allows the model to weigh
the importance of different input features relative to each other. It can be thought of as a method to compute a
weighted sum of values based on their relevance to a given query. Attention requires all the features to be mapped into
the $\vec{X} \in \R^{d_h \times n_\text{features}}$ subspace. Then, for all the features, the so-called
query, keys, and values are computed:

\begin{align}
    \vec{Q} & = \vec{W}_Q \vec{X}    \\
    \vec{K} & = \vec{W}_K \vec{X}    \\
    \vec{V} & = \vec{W}_V \vec{X}\,,
\end{align}

where $\vec{W}_Q, \vec{W}_K \in \R^{d_k \times d_h}, \vec{W}_V \in \R^{d_v \times d_h}$ are the trainable projection
matrices. The dimensionality of the query, key, and value vectors is $d_k, d_k, d_v$, respectively\footnote{We used $d_k
        = d_v = d_h$, which is a fairly standard practice.}.

The output of the whole layer is computed as:

\begin{equation}
    \Attention({\vec{Q}, \vec{K}, \vec{V}}) = \text{softmax}(\frac{\vec{Q}\vec{K}^T}{\sqrt{d_h}}) \vec{V} \,.
\end{equation}

First, the dot-product of the queries and keys is computed and scaled by $\frac{1}{\sqrt{d_k}}$. For large values of
$d_k$ this helps to avoid the softmax function from saturating. Then the softmax function is applied to obtain the
attention weights\footnote{Optionally, masking can be applied before softmax. For example in the \gls{nlp}, when
    predicting the next token, the model is only allowed to look at the previous tokens, so all the tokens after the current
    position are set to $-\infty$.}. Finally, the dot-product is computed with the values $\vec{V}$ to obtain the output of
the layer.

This mechanism enables the transformer to focus on different parts of the input data depending on the context provided
by the query. In practice, applying multiple such heads in parallel was found to be more effective\footnote{We used 4
    heads throughout all our experiments.}:

\begin{align}
    \MHA(\vec{Q}, \vec{K}, \vec{V}) & = \texttt{Concat}(\text{head}_1, \dots, \text{head}_h) \vec{W}_O \,,           \\
    \text{where head}_i             & = \Attention(\vec{Q} \vec{W}_i^Q, \vec{K} \vec{W}_i^K, \vec{V} \vec{W}_i^V)\,.
\end{align}

Here, queries, keys, and values, are projected by different matrices $\vec{W}_i^Q, \vec{W}_i^K, \vec{W}_i^V$ to each
head $i$, and then attention is applied in parallel. This allows the model to jointly attend to information from
different representation subspaces at different positions. Then, the outputs of the heads are concatenated and projected
to the desired dimensionality by the matrix $\vec{W}_O$.

The full \gls{ftt} transformer architecture is formalized as follows:

{\fontsize{11}{10}\selectfont
\begin{align}
    \label{eq:ftt}
    \FTT(x)                 & = \Linear(\LayerNorm(\FTTBlock(\dots(\FTTBlock(\Embed(x)))))_1) \\
    \FTTBlock(x)            & = \texttt{FeedForward}(x + \MHA(\LayerNorm(x)))                 \\
    \texttt{FeedForward}(x) & = x + \Dropout(\Linear(\GELU(\Linear(\LayerNorm(x))))) \,.
\end{align}
}

The architecture is composed of multiple blocks which are applied sequentially after mapping continuous and categorical
features into the embedding space. Each block consists of a \LayerNorm, \MHA, and two \Linear
layers. The residual connections are applied after the \MHA, and one more time after at the end of each block.
\LayerNorm layers are applied before each \MHA, and additionally before the first \Linear layer. After the first \Linear
layer, a \GELU activation is applied. Additionally, \Dropout is applied in the end of the block before the residual
connection. The output of the last block is passed to the \LayerNorm, and then the first token as taken and passed to the
final \Linear layer to obtain the log probabilities (logits) The whole structure is formalized in
\autoref{eq:ftt} as well as presented on the \autoref{fig:ftt}.


\begin{figure}[htb]
    \centering
    \includegraphics[width=0.5\textwidth]{figures/ftt.pdf} \caption{\gls{ftt} architecture.}
    \label{fig:ftt}
\end{figure}

% \subsubsection{Post-norm vs pre-norm formulation}

Since the original paper \cite{transformer}, not many things have changed with the transformer design. The notable
change is that \LayerNorm layer has been moved from after the \MHA and \texttt{FeedForward} layers
(post-norm formulation) to before them (pre-norm formulation). The pre-norm formulation has been shown to be more stable
and easier to train \cite{pre-norm}. We also adopt this change, following \cite{ft-transformer}.

To conclude, transformer is a powerful architecture. Attention mechanism essentially represent a form of the
computation and have shown to be able to approximate a wide range of functions really well. The possibility of
parallelization that comes with using modern \gls{gpu}s makes it possible to train such models very efficiently. The usage
of residual connections and \LayerNorm layers makes it possible to train very deep models.

In our experiments, we have observed that transformers have shown better regularization and better results. On the other
hand, training requires much more time and memory compared to the simple \glspl{resnet}, and produces a larger
$\text{CO}_2$ footprint.% The details about the training speeds are presented in the \appref{appendix:training-speed}.

% \subsubsection{Flatten last layers ?}

% \todo{TODO}

% Aside from the introduced handling of the missing/invalid values, there is one last difference from the original
% \gls{ftt}: instead of applying the \verb|MultiHeadAttention| layer in the end of the whole stack of blocks to obtain the
% logits, we use a fully-connected \verb|Linear| layer. I have no idea why I did this. The whole structure is shown on the
% \autoref{fig:ftt}.