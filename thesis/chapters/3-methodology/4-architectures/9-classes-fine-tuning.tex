\section{All classes vs signal and background only vs two-phase training}
\label{sec:binary}

In \cite{severin}, the authors experimented with both multiclass and binary classification formulations. The multiclass
formulation is a direct approach where the model is trained to differentiate among all classes simultaneously. In
contrast, the binary formulation is a specialized approach where the model is trained to distinguish only between signal
and background classes. All classes, except for the signal, are treated as background. This approach is motivated by the
hypothesis that the model can concentrate more of its resources on the primary task of signal and background
discrimination.

For \cite{severin}, based solely on the reported significance, the results were quite similar\footnote{Based solely on
    significance, it is challenging to conclude which approach is superior.} for both formulations. However, our
observations suggest that when the model is trained to differentiate among all classes, it demonstrates enhanced
learning capabilities\footnote{Based on accuracy, \gls{auc}, and significance.} and can potentially extract more
information from the input data (see \autoref{fig:two-phase-training}). This approach, however, also allocates resources
towards distinguishing individual background classes, which might not be essential for the primary task of signal and
background discrimination.

Consequently, we propose an alternative strategy, which we refer to as "two-phase training," to utilize the model's
resources more effectively. Initially, we train the model on all classes. Once satisfactory performance is achieved, we
transition to the binary formulation. In practical terms, this means we no longer penalize the model for misclassifying
among the background classes (e.g., if the true class is \ttw and the predicted one is \ttz, the prediction is still
deemed correct since both processes are background). This strategy allows the model to focus on the primary task of
signal and background discrimination after gleaning sufficient information from the distinctions among all classes. The
model first learns the patterns in the underlying physics comprehensively and then hones its focus on the primary task.

The results are summarized in \autoref{fig:two-phase-training}, showcasing the progress in AUC and accuracy for
differentiating signal from background. It's evident that the multiclass formulation outperforms the binary one, and
this performance is further enhanced when transitioning to the binary formulation after achieving good performance in
the multiclass phase.


\begin{figure}[htb]
    \centering
    \includegraphics[width=0.8\textwidth]{example-image-a}
    \caption{Comparison of the multiclass and binary formulations. The dashed line indicates the point where the model
        switches from the multiclass to the binary formulation. (second-phase-training)}
    \label{fig:two-phase-training}
\end{figure}