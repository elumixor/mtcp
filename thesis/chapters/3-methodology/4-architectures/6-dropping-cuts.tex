\section{Increasing statistics by dropping the cuts}

Our goal is to train a classifier, that would be able to perform well on the events inside our \gls{sr}. However, the
\gls{sr} is a very small region of the phase space, and the number of events in it is very small. This makes it
difficult to train a classifier that would be able to generalize well. In this section, we explore the possibility of
dropping the cuts and training the classifier on the space of all the events from the simulated dataset (except for the
validation/test sets). The reasoning behind this is that the classifier would be able to learn the structure of the
structure of events in the \gls{sr} from the events outside of it, as all share the same underlying physics, which would
be captured in the hidden layers of the network.

A somewhat similar approach was proposed by Nello and Simonetta, where they drop the PLIV cuts to obtain higher number
of samples for the $t\bar{t}$ events. They train a BDT on that extended dataset which achieves a better performance. In
order to do so, however, the re-weighting procedure is used to make the distributions the same as in SR.

When dropping all the cuts, such re-weighting should also be considered. However, we should also consider other factors
such as the class imbalance in terms of the raw (unweighted) events, and how it influences the final performance on the
SR. See section \ref{sec:weights} for more details.

\cite{tabular} explores why deep neural networks despite having shown a great performance on a variety of tasks such as
computer vision, natural language processing, and speech recognition, have not been widely adopted in the tabular data
domain. The main reason is that the tabular data is very sparse, and the number of samples is very small. This makes it
difficult to train a deep neural network which would generalize well. Random forests and gradient boosting methods
perform much better in this domain. However, as the number of samples increases, the performance of deep neural networks
improves and becomes comparable to the other methods.

The whole size of the simulated dataset contains 8.7M raw events. By applying the cuts, we are left with only 24K events
(0.3\%) inside the SR. After splitting this set further into 80\% training, and 20\% validation sets, we are left with
only 19K events for training. This is a very small number of samples to train a deep neural network on. Some classes
such as $t\bar{t}$ are very underrepresented in the training set, having as little as ~10 samples. This makes it
essentially impossible to train a classifier that would be able to generalize well. Furthermore, the weight associated
with these events is usually very high, which then results in a high loss, poor accuracy, etc. By extending the training
set to include all the events except for the validation/test sets, we increase the number of samples by a factor of
~400. This results in a much better performance, especially for the underrepresented classes such as $t\bar{t}$.
\autoref{fig:datasets} shows how the dataset is split into training, validation, and test sets.

\begin{figure}[htbp]
    \centering
    \includegraphics[width=0.5\textwidth]{figures/datasets.pdf}
    \caption{Splitting the dataset into training, validation, and test sets.}
    \label{fig:datasets}
\end{figure}