\subsection{Previous work}

In an attempt to further our understanding of the \tth process, this work builds upon the previous efforts made by
\severin and \jan. Their research also involved training feed-forward neural networks to distinguish between the \tth
and the background processes.

The primary architecture \cite{severin} utilized was a \gls{mlp}. The work examines the effect of various
hyperparameters such as the number of layers, embedding size, learning rate, and batch size on the significance. The
work also includes the estimation of statistical uncertainties, associated with the median signal strength. The authors
experimented with binary and multi-class classification, as well as proposed a staged network approach. The staged
network is composed of 5 binary classifiers (for \ttz, \ttw, \ttbar, $VV$, and all the other backgrounds grouped
together in one category "others"). Each of the classifiers is a \gls{mlp} itself, and during the training only receives
events of \tth and the corresponding background.

\todo{
    I fail to see how this would work in reality - how would we know which background to give to each binary classifier?
}

Although the staged network is its essence is equivalent to a single, larger \gls{mlp}, it allows for the training of
each sub-network on a \emph{different set of features}. This can potentially reduce the systematic uncertainties,
associated with the final prediction. \todo{although I don't understand really how...}

The highest mean significance obtained with a multi-class classifier was reported to be $Z = 3.064$ while the highest
mean significance obtained with a binary classifier was reported to be $Z = 3.114$. The expectation was that having the
\gls{nn} focus on differentiating between two classes only would improve the performance. However, the results show that
there was no significant difference. In our work, however, we have observed that when the classifier is trained on
multiple classes, the performance is increased (see \autoref{sec:binary}). The highest mean significance obtained with a
staged was reported to be $Z = 2.964$, which is very similar to the other results.





In \jan, the experiments were also extended beyond simple \glspl{mlp}, experimenting with TabNet \cite{tabnet} and
XGBoost \cite{xgboost}. The authors have also experimented with different hyperparameters, as well as different
fractions of the training set, and different feature sets. The best results were obtained with XGBoost trained on all
the features and the whole training set, which is unsurprising. The highest mean significance obtained with XGBoost is
reported as $Z = 2.90$. The authors have also evaluated the uncertainty, associated with the prediction, which is
reported as $\mu = 1 + 0.42 /- 0.37$.






Our research seeks to improve upon these previous efforts by introducing more complex architectures and advanced
optimization techniques, which will be discussed in the following sections.

\subsection[MLP]{\gls{mlp}}

In previous work, \severin utilized \glspl{mlp} as the primary model architecture.
While he experimented with combining multiple \glspl{mlp}, this approach is essentially equivalent to using a single,
larger \gls{mlp}. This can be formalized as in \cite{ft-transformer}:

\begin{align}
    \texttt{MLP(x)}      & = \Linear(\texttt{MLPBlock}(\dots(\texttt{MLPBlock}(x)))) \\
    \texttt{MLPBlock(x)} & = \Dropout(\sigma(\Linear(x)))
\end{align}

Where $\sigma$ is the activation function\footnote{Common activation functions include ReLU, Leaky ReLU (LReLU)
    \cite{relu, lrelu}, sigmoid, $\tanh$, and others. Throughout our experiments we mostly use GELU activation
    \cite{gelu}.},
$\Linear$ is a linear transformations:

\begin{equation}
    \Linear(x) = Wx + b\,,
\end{equation}

and $\Dropout$ is a layer that randomly (with a fixed probability $p$) sets a fraction of the input features to
zero \cite{dropout}:

\begin{equation}
    \Dropout_i(\vec{x}) = \begin{cases}
        0,         & \text{with probability } p \\
        \vec{x}_i, & \text{otherwise}
    \end{cases}
\end{equation}