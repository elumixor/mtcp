\section{Effect of the reduced training set size}

Our extended training set contains roughly 8.7 million events. Combined with the fact that our \gls{nn} is quite large
as well (about 4 million learnable parameters), the training takes a long time. We have thus decided to investigate
the effect of the reduced training set size on the performance of the classifier, hoping that we can reduce the
training time without sacrificing too much performance. The results are summarized on the \autoref{tab:reduced-training-set}.

\begin{figure}[htb]
    \centering
    \includegraphics[width=\textwidth]{figures/ml/fraction/results.pdf}
    \caption[Effect of reduced training set size on classifier performance]
    {Effect of reduced training set size on classifier performance. \gls{ftt} with 2 blocks was used.}
    \label{tab:reduced-training-set}
\end{figure}

We have performed 4 experiments with 100\%, 50\%, 10\%, and 5\% of the extended training set \ttrn. The results are
shown in \autoref{tab:reduced-training-set}. During each experiment, we have kept the same random seed to keep the
initialization of the \gls{nn} the same. We also kept the same batch size and learning rate. The results show that the
performance of the classifier does not degrade significantly from 100\% and 50\%. Such training set size of about 4
million events seem adequate to obtain some sufficiently good results. Of course, for maximum performance, it is best to
use the full training set. When dropping to 10\% or 5\% of the training set, the performance degrades significantly,
but is nevertheless much better than when the standard training set \ttrn is used.
