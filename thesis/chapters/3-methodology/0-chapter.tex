% This chapter describes the methodology used in this study. It includes the formulation of the problem, the evaluation metrics used, and the architectures tested.
\chapter{Methodology}
\label{ch:Methodology}

% Include the formulation section
\section{Problem Formulation}

Our primary objective is to distinguish the ttH events from other processes (or events) produced in the LHC. This
separation task can be naturally formulated as a classification problem. In the broad field of machine learning,
classification is the task of predicting the discrete class label of an observation given its features. In our case, the
observations are the events and the class labels represent different processes, including the ttH process.

Classification problems can further be classified into binary and multi-class, depending on the number of classes.
Binary classification, as the name suggests, involves two classes. On the other hand, multi-class classification refers
to problems where an observation can belong to one of more than two classes. Depending on the nature and distribution of
the data, one may choose either binary or multi-class classification. In our case, we can approach the task either as a
binary problem (ttH vs not ttH) or as a multi-class problem (differentiating among various processes).

As with many machine learning tasks, we approach this as a supervised learning problem. In supervised learning, we have
a set of labeled observations (known as the training data) and the goal is to learn a function that maps the features to
the labels. Once this function is learned, it can be used to predict the labels (in our case, process classes) of new,
unseen observations.

Given a dataset with $N$ observations, we have each observation $i$ represented as a tuple $(x_i, y_i, w_i)$, where
$x_i$ is a feature vector (a representation of our event), $y_i$ is a class label, and $w_i$ is a weight associated with
the observation. The objective is to learn a function $f$ such that given a new observation $x$, we can predict its
class label $y$.

In the case of binary classification, $y_i$ can take on values 0 (for background) or 1 (for signal), while for
multi-class classification, $y_i$ can take on values from $\{0, 1, ..., C\}$ where $C$ is the number of classes
(representing different processes).

The learning of the function $f$ is driven by the minimization of a loss function $L$ that quantifies the discrepancy
between the predicted and true labels. For classification problems, a common choice is the cross-entropy loss function.

For a binary classification, the cross-entropy loss for a single observation is given by:

$$ L(y_i, f(x_i)) = -y_i \log(f(x_i)) - (1 - y_i) \log(1 - f(x_i)) $$

where $f(x_i)$ is the probability of the observation $i$ being of class 1. When you sum up the losses for all
observations and normalize by the total number of observations, you get the average loss for the dataset, also known as
the empirical risk:

$$ R(f) = -\frac{1}{N}\sum_{i=1}^{N} y_i \log(f(x_i)) + (1 - y_i) \log(1 - f(x_i)) $$

In the case of a weighted dataset, the empirical risk becomes:

$$ R(f) = -\frac{1}{N}\sum_{i=1}^{N} w_i \big[ y_i \log(f(x_i)) + (1 - y_i) \log(1 - f(x_i)) \big] $$

For multi-class classification, the cross-entropy loss can be extended to accommodate more than two classes. This is
usually done by using one-hot encoding for the class labels and applying a softmax function to the output of the
function $f$. The loss for a single observation then becomes:

$$ L(y_i, f(x_i)) = - \sum_{c=1}^{C} y_{i,c} \log(f_{i,c}(x_i)) $$

where $y_{i,c}$ is the true label (0 or 1) of observation $i$ for class $c$, and $f_{i,c}(x_i)$ is the predicted
probability of observation $i$ belonging to class $c$.

As before, the empirical risk is the sum of the losses for all observations, and in the case of a weighted dataset, it
is defined as:

$$ R(f) = -\frac{1}{N}\sum_{i=1}^{N} w_i \sum_{c=1}^{C} y_{i,c} \log(f_{i,c}(x_i)) $$

In this context, the function $f$ that minimizes this empirical risk is the solution to our classification problem.

% Include the evaluation section
\section{Evaluating the Classifier Performance}

The performance of a classifier is not solely determined by its ability to make correct predictions. There are several
measures of performance, each emphasizing a different aspect of the classifier's behavior. Choosing the right measure of
performance is vital as it directly impacts the optimization procedure during training and influences how well the model
generalizes to unseen data. Therefore, it is important to understand these measures in detail.

\subsection{Loss Function: The Objective of Training}

The loss function quantifies the discrepancy between the model's predictions and the actual targets in the training set.
It provides an objective measure that we strive to minimize during the training process. Cross-entropy loss, also known
as log loss, is a commonly used loss function for classification tasks, including ours:

\begin{equation}
    L(\mathbf{y}, \mathbf{\hat{y}}) = -\frac{1}{N}\sum_{i=1}^{N}\sum_{j=1}^{M} y_{ij} \log(\hat{y}_{ij})
\end{equation}

where $N$ is the number of samples, $M$ is the number of classes, $\mathbf{y}$ is the true label, and $\mathbf{\hat{y}}$
is the predicted label.

\subsection{Validation and Test Sets: Generalization Checkpoints}

\begin{figure}[htbp]
    \centering
    \includegraphics[width=0.8\textwidth]{figures/plots/ml/losses.png}
    \caption{Training and validation losses during the training process. \gls{ftt} with 2 blocks and embedding size
        of 256 was used on the plot (see \autoref{sec:ftt}). Around the 30th epoch, the training loss starts to
        decrease while the validation loss starts to increase, indicating that the model is overfitting to
        the training data. It is at this point that the checkpoint is saved and used for evaluation.}
    \label{fig:losses}
\end{figure}

To assess the generalization ability of the classifier, the dataset is typically divided into three subsets: training,
validation, and test sets. The training set is used to adjust the model's parameters, the validation set is used to tune
the hyperparameters and provide an unbiased evaluation of the model during training, and the test set is used to assess
the performance of the fully-trained model.

During training, not only is the training loss monitored, but also the validation loss. The validation loss provides an
unbiased estimate of the model's performance on unseen data and is crucial for preventing overfitting. If the training
loss continues to decrease while the validation loss starts to increase, the model is likely overfitting to the training
data and failing to generalize to unseen data (see \autoref{fig:losses}). The point at which the validation loss is
minimized is typically chosen as the stopping point for training (early stopping).

\subsubsection{Performance Metrics: More than Accuracy}

While the loss function provides a measure of the classifier's performance, it does not tell the whole story. For
instance, the loss function does not capture the trade-off between correctly identifying positive instances
(sensitivity) and correctly identifying negative instances (specificity). To address this, several other performance
metrics are commonly used to evaluate classifiers.

\subsubsection{Confusion Matrix: A Comprehensive Performance Snapshot}

The confusion matrix provides a comprehensive view of the classifier's performance. For a binary classification task, it
is a 2x2 matrix where the rows correspond to the true classes and the columns correspond to the predicted classes:

\begin{equation}
    \begin{pmatrix}
        \text{TP} & \text{FP} \\
        \text{FN} & \text{TN} \\
    \end{pmatrix}
\end{equation}

where TP (true positive) is the number of positive instances correctly identified as positive, TN (true negative) is the
number of negative instances correctly identified as negative, FP (false positive) is the number of negative instances
incorrectly identified as positive (Type I error), and FN (false negative) is the number of positive instances
incorrectly identified as negative (Type II error).

In our context, the confusion matrix can be extended to a multi-class scenario, leading to an MxM matrix for an M-class
classification task. Moreover, to reflect the underlying data distribution, the confusion matrix should be weighted,
with each instance contributing to the confusion matrix proportional to its weight.

The confusion matrix serves as the basis for several other performance metrics, including accuracy, F1 score, and area
under the ROC curve (AUC-ROC).

\subsubsection{Weighted Confusion Matrix: Accounting for the Data Distribution}

When using machine learning in high-energy physics, the data distribution used to train the model often differs from the
real-world data distribution we aim to make predictions on. This is due to the Monte Carlo simulations we use to produce
our training data. While these simulations are designed to model the physical processes as accurately as possible, they
do not perfectly represent the real data distribution.

This discrepancy is resolved by attaching a weight to each event in the simulation. The weight represents the ratio of
the number of expected events in the real data to the number of events in the simulated data. Using these weights, we
can then scale the distribution of the simulated data to match the expected distribution of the real data.

When evaluating the classifier's performance, it is crucial to use a weighted confusion matrix to account for these
event weights. In the weighted confusion matrix, each instance contributes to the matrix proportional to its weight. The
weighted confusion matrix for a binary classification task is given by:

\begin{equation}
    \begin{pmatrix}
        \text{TP}_{w} & \text{FP}_{w} \\
        \text{FN}_{w} & \text{TN}_{w} \\
    \end{pmatrix}
\end{equation}

where each term (TP, FP, FN, TN) is the sum of the weights of the corresponding instances.

The performance metrics derived from this weighted confusion matrix (accuracy, F1 score, AUC-ROC, etc.) then reflect the
performance of the classifier on the actual data distribution we are interested in. Hence, using a weighted confusion
matrix enables us to evaluate and optimize our classifier in a manner that is directly relevant to our ultimate goal:
the accurate classification of ttH events in the actual LHC data.

\subsubsection{Accuracy: The Proportion of Correct Predictions}

Accuracy is the most intuitive performance metric. It is the proportion of total predictions that are correct and can be
calculated directly from the confusion matrix:

\begin{equation}
    \text{Accuracy} = \frac{TP + TN}{TP + FP + FN + TN}
\end{equation}

\subsubsection{F1 Score: The Balance Between Precision and Recall}

The F1 score is the harmonic mean of precision and recall, providing a balance between the two. Precision (also known as
positive predictive value) is the proportion of positive identifications that were actually correct, while recall (also
known as sensitivity) is the proportion of actual positives that were identified correctly. The F1 score can be
calculated as:

\begin{equation}
    \text{F1 Score} = 2 \cdot \frac{\text{Precision} \cdot \text{Recall}}{\text{Precision} + \text{Recall}}
\end{equation}

\subsubsection{ROC Curve and AUC: The Trade-off Between Sensitivity and Specificity}

The receiver operating characteristic (ROC) curve is a plot of the true positive rate (recall or sensitivity) against
the false positive rate (1 - specificity) for different classification thresholds. The area under the ROC curve
(AUC-ROC) measures the classifier's ability to distinguish between classes. A perfect classifier has an AUC-ROC of 1,
while a random classifier has an AUC-ROC of 0.5.

These metrics, combined with the loss function, provide a comprehensive view of the classifier's performance and guide
the optimization process during training. They also provide a robust measure for comparing different classifiers or the
same classifier with different hyperparameters.

% Include the architectures section
\section{V8 adaptation}

\section{\gls{sr} cut expression}

 {\scriptsize
  \begin{verbatim}
    custTrigMatch_LooseID_FCLooseIso_DLT
    && (dilep_type && (lep_ID_0*lep_ID_1)>0)
    && ((lep_Pt_0 >= 10e3 && lep_Pt_1 >= 10e3) && (fabs(lep_Eta_0) <= 2.5 && fabs(lep_Eta_1) <= 2.5)
        && ((abs(lep_ID_0) == 13 && lep_isMedium_0 && lep_isolationLoose_VarRad_0 && passPLIVTight_0)
            || ((abs(lep_ID_0) == 11 && lep_isTightLH_0 && lep_isolationLoose_VarRad_0 && passPLIVTight_0
                && lep_ambiguityType_0 == 0 && lep_chargeIDBDTResult_recalc_rel207_tight_0 > 0.7)
                && ((!(!(lep_Mtrktrk_atConvV_CO_0 < 0.1 && lep_Mtrktrk_atConvV_CO_0 >= 0 && lep_RadiusCO_0 > 20)
                    && (lep_Mtrktrk_atPV_CO_0 < 0.1 && lep_Mtrktrk_atPV_CO_0 >= 0)))
                    && !(lep_Mtrktrk_atConvV_CO_0 <0.1 && lep_Mtrktrk_atConvV_CO_0 >= 0 && lep_RadiusCO_0 > 20))))
            && ((abs(lep_ID_1) == 13 && lep_isMedium_1 && lep_isolationLoose_VarRad_1 && passPLIVTight_1)
                || ((abs(lep_ID_1) == 11 && lep_isTightLH_1 && lep_isolationLoose_VarRad_1 && passPLIVTight_1
                    && lep_ambiguityType_1 == 0 && lep_chargeIDBDTResult_recalc_rel207_tight_1 > 0.7)
                    && ((!(!(lep_Mtrktrk_atConvV_CO_1 < 0.1 && lep_Mtrktrk_atConvV_CO_1 >= 0 && lep_RadiusCO_1 > 20)
                        && (lep_Mtrktrk_atPV_CO_1 < 0.1 && lep_Mtrktrk_atPV_CO_1 >= 0)))
                        && !(lep_Mtrktrk_atConvV_CO_1 < 0.1 && lep_Mtrktrk_atConvV_CO_1 >= 0 && lep_RadiusCO_1 > 20)))))
    && nTaus_OR==1
    && nJets_OR_DL1r_85>=1
    && nJets_OR>=4
    && ((dilep_type==2) || abs(Mll01-91.2e3)>10e3)
\end{verbatim}
 }

We have kept the cuts the same as \cite{severin}, except for the cut on the \verb|nJets_OR| to \verb|>=4| to keep
consistent definition \gls{sr} definition across the group \todo{refer to the BDT group - how?}.

\section{Yields Plots}
\label{appendix:yields}

\begin{figure}[htb!]
    \centering
    \begin{subfigure}{0.45\textwidth}
        \includegraphics[width=\linewidth]{figures/yields/lep-pt-0.pdf}
        \caption{Distribution of the transverse momentum of the leading lepton.}
    \end{subfigure}\hfill%
    \begin{subfigure}{0.45\textwidth}
        \includegraphics[width=\linewidth]{figures/yields/lep-pt-1.pdf}
        \caption{Distribution of the transverse momentum of the subleading lepton.}
    \end{subfigure}
\end{figure}

\begin{figure}[htb!]
    \centering
    \begin{subfigure}{0.45\textwidth}
        \includegraphics[width=\linewidth]{figures/yields/n-jets.pdf}
        \caption{Distribution of the number of jets.}
    \end{subfigure}\hfill%
    \begin{subfigure}{0.45\textwidth}
        \includegraphics[width=\linewidth]{figures/yields/n-bjets.pdf}
        \caption{Distribution of the number of $b$-jets.}
    \end{subfigure}
\end{figure}

\begin{figure}[htb!]
    \centering
    \begin{subfigure}{0.45\textwidth}
        \includegraphics[width=\linewidth]{figures/yields/tau-width.pdf}
        \caption{Distribution of the $\tau$-jet width.}
    \end{subfigure}\hfill%
\end{figure}
\input{chapters/3-data/v8/2-samples.tex}
\input{chapters/3-data/v8/3-distributions.tex}
\clearpage \subsection{Features used}

\subsubsection{\texttt{lep\_E\_X}} The energy of the Xth lepton.

\subsubsection{\texttt{DRjj\_lead}} $\Delta R$ between the two leading jets. $\Delta R$ is a distance metric in the
$\eta-\phi$ space frequently used in particle physics.

\subsubsection{\texttt{Ptll01}} The transverse momentum of the dilepton system made up of the two leading leptons.

\subsubsection{\texttt{lep\_nTrackParticles\_X}} The number of track particles associated with the Xth lepton.

\subsubsection{\texttt{custTrigMatch\_LooseID\_FCLooseIso\_DLT}} Custom trigger matching for loosely identified and loosely
isolated leptons, likely related to the dilepton trigger

\subsubsection{\texttt{Mll01}} The invariant mass of the two leading leptons.

\subsubsection{\texttt{Mlll012}} The invariant mass of the three leading leptons.

\subsubsection{\texttt{total\_charge}} The sum of the electric charges of the particles in the event.

\subsubsection{\texttt{HT}} The scalar sum of the transverse momenta of all jets in the event.

\subsubsection{\texttt{HT\_lep}} The scalar sum of the transverse momenta of all leptons in the event.

\subsubsection{\texttt{lep\_Eta\_X}} The pseudorapidity of the Xth lepton.

\subsubsection{\texttt{nTaus\_OR\_Pt25}} The number of overlapping-removed taus with a transverse momentum above 25 GeV.

\subsubsection{\texttt{nFwdJets\_OR}} The number of overlapping-removed forward jets.

\subsubsection{\texttt{MLepMet}} The invariant mass of a lepton and the missing transverse energy vector.

\subsubsection{\texttt{taus\_DL1r\_X}} The DL1r score for the Xth tau.

\subsubsection{\texttt{lep\_isolationLoose\_VarRad\_X}} Indicates whether a lepton (where X refers to the lepton index)
passes an isolation cut with a variable radius. Looser isolation cuts allow more nearby activity in the detector.

\subsubsection{\texttt{lep\_EtaBE2\_X}} The pseudorapidity of the Xth lepton in the second layer of the electromagnetic
calorimeter.

\subsubsection{\texttt{HT\_fwdJets}} The scalar sum of the transverse momenta of all forward jets in the event.

\subsubsection{\texttt{taus\_width\_X}} The width of the Xth tau.

\subsubsection{\texttt{nJets\_OR\_DL1r\_85}} Count of jets that pass overlap removal (OR) and are b-tagged according to the
DL1r algorithm at the 85\% working point.

\subsubsection{\texttt{lep\_nInnerPix\_X}} Number of hits in the inner pixel detector associated with the lepton, where X
refers to the lepton index.

\subsubsection{\texttt{met\_phi}} The azimuthal angle of the missing transverse energy in the event.

\subsubsection{\texttt{DeltaR\_max\_lep\_bjet77}} The maximum DeltaR value between a lepton and a b-tagged jet. The "77"
may refer to the working point of the b-tagging algorithm.

\subsubsection{\texttt{MbX}} Invariant mass associated with the leading b-jet in the event

\subsubsection{\texttt{lep\_RadiusCO\_X}} Possibly the radius of the cone used for isolation of the lepton, or
alternatively a parameter associated with the trajectory of the lepton.

\subsubsection{\texttt{lep\_Mtrktrk\_atConvV\_CO\_X}} The invariant mass of track pairs at the conversion vertex for lepton
X. This might be related to photon conversions into an electron-positron pair.

\subsubsection{\texttt{lep\_Z0SinTheta\_X}} The z0 impact parameter times the sine of the lepton's polar angle.

\subsubsection{\texttt{lep\_Pt\_X}} The transverse momentum of the Xth lepton.

\subsubsection{\texttt{mjjMax\_frwdJet}} The maximum invariant mass of a pair of forward jets.

\subsubsection{\texttt{dilep\_type}} The type of dilepton event (e.g., $ee$, $\mu e$, $\mu \mu$).

\subsubsection{\texttt{eta\_frwdjet}} The pseudorapidity of the forward jet.

\subsubsection{\texttt{Mlb}} Invariant mass of a lepton and a b-jet.

\subsubsection{\texttt{taus\_RNNJetScoreSigTrans\_X}} Transformed RNN-based score for tau lepton, possibly to better
separate signal from background.

\subsubsection{\texttt{minDeltaR\_LJ\_X}} The minimum $\Delta R$ distance between the Xth lepton and any jet in the event.

\subsubsection{\texttt{nTaus\_OR}} Number of tau leptons that pass overlap removal. Overlap removal is a step in particle
reconstruction where, for instance, an object identified as both a jet and a tau would be considered only as one or the
other.

\subsubsection{\texttt{DeltaR\_min\_lep\_jet}} The minimum $\Delta R$ distance between a lepton and a jet in the event.

\subsubsection{\texttt{lep\_sigd0PV\_X}} Significance of the transverse impact parameter (d0) of the lepton X with respect
to the primary vertex (PV). This is a common variable for distinguishing prompt particles produced in the primary
collision from secondary particles produced in a decay.

\subsubsection{\texttt{taus\_eta\_X}} The pseudorapidity of the Xth tau.

\subsubsection{\texttt{HT\_jets}} The scalar sum of the transverse momenta of all jets (not forward jets) in the event.

\subsubsection{\texttt{lep\_Phi\_X}} The azimuthal angle (in radians) of the Xth lepton.

\subsubsection{\texttt{bTagSF\_weight\_DL1r\_85}} A weight applied to events based on the scale factor for b-tagging using
the DL1r algorithm at an 85\% efficiency working point. This scale factor corrects the b-tagging efficiency in Monte
Carlo simulations to match that observed in real data.

\subsubsection{\texttt{lep\_chargeIDBDTResult\_recalc\_rel207\_tight\_X}} The outcome of a BDT-based charge identification
for a lepton, recalculated with some specific settings, and applying a 'tight' threshold.

\subsubsection{\texttt{taus\_phi\_X}} The azimuthal angle (in radians) of the Xth tau.

\subsubsection{\texttt{taus\_passJVT\_X}} A boolean flag indicating whether the Xth tau passes the jet vertex tightness
(JVT) requirement.

\subsubsection{\texttt{jets\_eta}} The pseudorapidity of the jets (array).

\subsubsection{\texttt{taus\_charge\_X}} The charge of the Xth tau.

\subsubsection{\texttt{passPLIVTight\_X}} Boolean flag indicating if a lepton with high transverse momentum passes the
"tight" criteria of the Prompt Lepton Veto (PLIV), a tool for identifying non-prompt light leptons.

\subsubsection{\texttt{lep\_Mtrktrk\_atPV\_CO\_X}} The invariant mass of track pairs at the primary vertex for lepton X.
This could be related to certain types of particle decays happening at the primary collision vertex.

\subsubsection{\texttt{taus\_JetRNNSigMedium\_X}} RNN-based score for tau lepton, used to distinguish tau leptons from
jets, with 'medium' selection criteria.

\subsubsection{\texttt{minOSMll}} The minimum invariant mass of oppositely-signed dilepton pairs.

\subsubsection{\texttt{lep\_ID\_X}} The identification number for the Xth lepton.

\subsubsection{\texttt{Mllll0123}} The invariant mass of the four leading leptons.

\subsubsection{\texttt{custTrigSF\_TightElMediumMuID\_FCLooseIso\_DLT}} Custom trigger scale factor, for events with a
tight electron and a medium muon, both of which are loosely isolated, likely related to the dilepton trigger (DLT).

\subsubsection{\texttt{best\_Z\_Mll}} The invariant mass of the dilepton system that is closest to the Z boson mass.

\subsubsection{\texttt{met\_met}} The missing transverse energy in the event.

\subsubsection{\texttt{MtLep1Met}} Transverse mass between the leading lepton and missing transverse energy. Transverse
mass is often used in searches for particles that decay to a lepton and a neutrino.

\subsubsection{\texttt{lep\_ambiguityType\_X}} Type of ambiguity for lepton identification, where X refers to the lepton
index. Ambiguity could arise from several factors, such as a single track matching with multiple reconstructed
particles.

\subsubsection{\texttt{jets\_phi}} The azimuthal angle (in radians) of the jets (array).

\subsubsection{\texttt{lep\_isMedium\_X}} Boolean flag indicating if a lepton passes the 'medium' selection criteria.

\subsubsection{\texttt{taus\_RNNJetScore\_X}} RNN-based score for tau lepton, used to distinguish tau leptons from jets.

\subsubsection{\texttt{MtLepMet}} The transverse mass of a lepton and the missing transverse energy vector.

\subsubsection{\texttt{DeltaR\_min\_lep\_jet\_fwd}} The minimum $\Delta R$ distance between a lepton and a forward jet in the event.

\subsubsection{\texttt{jets\_e}} The energy of the jets (array).

\subsubsection{\texttt{minOSSFMll}} The minimum invariant mass of oppositely-signed, same-flavor dilepton pairs.

\subsubsection{\texttt{nJets\_OR}} The number of overlapping-removed jets.

\subsubsection{\texttt{total\_leptons}} The total number of leptons in the event.

\subsubsection{\texttt{taus\_numTrack\_X}} The number of tracks associated with the Xth tau.

\subsubsection{\texttt{HT\_taus}} Scalar sum of the transverse momenta ($P_t$) of all tau leptons in the event.

\subsubsection{\texttt{taus\_passEleOLR\_X}} A boolean flag indicating whether the Xth tau passes the electron overlap
removal.

\subsubsection{\texttt{HT\_inclFwdJets}} The scalar sum of the transverse momenta of all jets, including forward jets, in
the event.

\subsubsection{\texttt{DRll01}} The $\Delta R$ distance between the two leading leptons.

\subsubsection{\texttt{taus\_JetRNNSigLoose\_X}} RNN-based score for tau lepton, used to distinguish tau leptons from
jets, with 'loose' selection criteria.

\subsubsection{\texttt{taus\_pt\_X}} The transverse momentum of the Xth tau.

\subsubsection{\texttt{bTagSF\_weight\_DL1r\_77}} A weight applied to events based on the scale factor for b-tagging using
the DL1r algorithm at an 77\% efficiency working point. This scale factor corrects the b-tagging efficiency in Monte
Carlo simulations to match that observed in real data.

\subsubsection{\texttt{flag\_JetCleaning\_LooseBad}} A flag variable indicating whether a jet passes a loose cleaning cut
to remove bad or noisy jets from the analysis.

\subsubsection{\texttt{taus\_fromPV\_X}} A boolean flag indicating whether the Xth tau comes from the primary vertex.

\subsubsection{\texttt{best\_Z\_other\_MtLepMet}} The transverse mass between the lepton and missing transverse energy for
the event that best reconstructs a Z boson using other criteria.

\subsubsection{\texttt{nJets\_OR\_DL1r\_77}} Count of jets that pass overlap removal (OR) and are b-tagged according to the
DL1r algorithm at the 77\% working point.

\subsubsection{\texttt{jets\_pt}} The transverse momentum of the jets (array).

\subsubsection{\texttt{lep\_isTightLH\_X}} Boolean flag indicating if a lepton passes the 'tight' Likelihood-based
identification criteria.

\subsubsection{\texttt{taus\_JetRNNSigTight\_X}} RNN-based score for tau lepton, used to distinguish tau leptons from
jets, with 'tight' selection criteria.

\subsubsection{\texttt{sumPsbtag}} The sum of b-tagging weights for jets in the event.

\subsubsection{\texttt{taus\_decayMode\_X}} The decay mode of the Xth tau.

\subsubsection{\texttt{dEta\_maxMjj\_frwdjet}} The maximum difference in pseudorapidity ($\eta$) between two forward jets.

\subsubsection{\texttt{max\_eta}} The maximum pseudorapidity among all particles in the event.

\subsubsection{\texttt{best\_Z\_other\_Mll}} The invariant mass of the dilepton system that is closest to the Z boson mass,
not considering the leading leptons.

\subsubsection{\texttt{taus\_passEleBDT\_X}} Flag indicating if a tau lepton passes the Electron Boosted Decision Tree
discriminator.

\begin{figure}[hbtp]
    \centering
    \includegraphics[width=\textwidth]{figures/ml/features/top20.pdf}
    \caption{Feature importance for the top 20 most important features. Feature importance was calculated using the
        \gls{ig} method \cite{ig}.}
    \label{fig:feature_importance}
\end{figure}

\clearpage

% Generate the results figure
\begin{figure}
    \centering
    \includegraphics[width=\textwidth]{figures/ml/results/results.pdf}
    \caption{Results of the different architectures and configurations. FTT refers to the \gls{ftt}.}
    \label{tab:results}
\end{figure}

% \begin{figure}
%     \centering
%     \includegraphics[width=\textwidth]{figures/ml/results/acc_multi.pdf}
%     \caption{Accuracy when separating only between signal and background.}
% \end{figure}

% \begin{figure}
%     \centering
%     \includegraphics[width=\textwidth]{figures/ml/results/auc_mean.pdf}
%     \caption{Accuracy when separating only between signal and background.}
% \end{figure}

% \begin{figure}
%     \centering
%     \includegraphics[width=\textwidth]{figures/ml/results/auc_tth.pdf}
%     \caption{Accuracy when separating only between signal and background.}
% \end{figure}

% \begin{figure}
%     \centering
%     \includegraphics[width=\textwidth]{figures/ml/results/f1.pdf}
%     \caption{Accuracy when separating only between signal and background.}
% \end{figure}

% \begin{figure}
%     \centering
%     \includegraphics[width=\textwidth]{figures/ml/results/sig.pdf}
%     \caption{Accuracy when separating only between signal and background.}
% \end{figure}

% \begin{figure}
%     \centering
%     \includegraphics[width=\textwidth]{figures/ml/results/loss.pdf}
%     \caption{Accuracy when separating only between signal and background.}
% \end{figure}
