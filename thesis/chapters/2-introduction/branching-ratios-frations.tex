\subsection{Branching Ratios}

A branching ratio (or branching fraction) in particle physics is a measure of the fraction of particles that decay via a
particular decay mode with respect to the total decay modes. It is a critical parameter in understanding particle
interactions, as it gives insight into the relative probabilities of various outcomes of a decay process.

For example, let us consider the production of a top-antitop pair along with a Higgs boson in a proton-proton collision,
also referred to as the \tth process. The Higgs boson can decay into different particles, each with a certain branching
ratio. In the case of the Higgs boson decaying into a bottom-antibottom quark pair, the branching ratio is approximately
58\%, which is the highest branching ratio among all the Higgs decay channels.

However, the probability of producing a Higgs boson, along with a top-antitop pair, in a proton-proton collision is very
small, estimated to be about 1\% of the total cross-section (\autoref{sec:lss})
\footnote{\url{https://home.cern/news/news/physics/higgs-boson-comes-out-top}}
\footnote{\url{https://home.cern/news/press-release/cern/higgs-boson-reveals-its-affinity-top-quark}}. This small
percentage, coupled with the fact that the \gls{lhc} produces millions of collisions per second, means that a
substantial amount of data must be
analyzed to isolate and identify the relatively few \tth events.

The ability to isolate these rare events depends not only on the branching ratios of the Higgs boson and the top quarks,
but also on our ability to accurately detect and identify the decay products. For instance, in the case of the Higgs
boson decaying into a bottom-antibottom pair, we need to be able to identify these 'b-jets' among a plethora of other
particles. This is a major challenge in particle physics and is a central focus of this thesis.