\section{\tth Process}

The \tth process is a specific type of event that occurs within the \gls{lhc}'s high-energy
particle collisions. In this process, two gluons collide, resulting in a pair of top quarks and a Higgs boson.

The \tth process is of particular interest in particle physics due to its potential to shed light on the nature of the
Higgs boson. The Higgs boson is pivotal in the \gls{sm} of particle physics, as it is associated with the Higgs
field, which gives other particles their mass. Despite its significance, the Higgs boson remains one of the least
understood particles due to its elusive nature and the difficulty involved in its detection.

In a \tth event, the Higgs boson quickly decays into other particles, making it impossible to observe it directly. Instead,
physicists must reconstruct its presence from the particles into which it decays. The top quarks, on the other hand,
have a higher lifetime, making them easier to detect. The \tth process is unique in that it allows the observation of the
Higgs boson's interactions with top quarks - the heaviest known fundamental particle - thereby providing a direct probe
into the Higgs mechanism.

The primary goal of this research is to separate \tth process events from other events, which is a challenging task given
the similarities between \tth events and certain background processes. By leveraging \gls{dl} techniques, this task can
be achieved more effectively, thereby facilitating a more detailed study of the Higgs boson and its properties.

\subsection{\lss channel}
\label{sec:lss}

% ttH
\begin{figure}[htb]
    \centering
    \includegraphics[width=0.5\textwidth]{figures/feynman/2lss1tau.pdf}
    \caption{Feynman diagram of the \lss process.}
    \label{fig:2lss1tau}
\end{figure}

The work at \gls{cern} is often divided into different channels, with each channel focusing on a specific final state of
interest. These channels are orthogonal, meaning that each event can only belong to one channel, preventing any overlap
in the analysis.

One such channel, and the main focus of this thesis, is the two-lepton same-sign plus one tau (\lss) channel. This
specific final state arises from the decay of a top-antitop-Higgs (\tth) system, where one top quark decays to a $W$ boson
and a $b$ quark, with the $W$ boson further decaying to a lepton and a neutrino. The other top quark also decays to a $W$
boson and a $b$ quark, but in this case, the $W$ boson decays into a pair of quarks. Lastly, the Higgs boson decays to a
pair of tau leptons, where one tau lepton decays into another lepton and two neutrinos, while the other tau lepton
decays to a pair of quarks and a neutrino. This complex series of decays results in a final state consisting of two
same-sign leptons and a hadronically decaying tau lepton, hence the name \lss channel (as shown in \autoref{fig:2lss1tau}).

The branching ratios for these decays are as follows:

\begin{align*}
    \text{Higgs boson to a pair of tau leptons: }\mathcal{B}(H \rightarrow \tau\tau)    & = 6\%   \\
    \text{Tau lepton to a $W$ boson and a neutrino: }\mathcal{B}(\tau \rightarrow W\nu) & = 100\% \\
    \text{Top quark to a $W$ boson and a b quark: }\mathcal{B}(t \rightarrow Wb)        & = 100\% \\
    \text{$W$ boson to a lepton and a neutrino: }\mathcal{B}(W \rightarrow l\nu)        & = 33\%  \\
    \text{$W$ boson to a pair of quarks: }\mathcal{B}(W \rightarrow qq)                 & = 67\%
\end{align*}

Given that the theoretical cross-section for the \tth process is $\sigma_\tth = 500 \text{ fb}$. The luminosity for the
measurement period (3 runs) is ${\mathcal{L} = 140 \text{ fb}^{-1}}$, the total number of expected \tth events is
approximately ${N_\tth = \sigma_\tth \cdot \mathcal{L} = 70\ 000}$.  When accounting for the aforementioned branching
ratios, the expected number of \tth events in the \lss final state is reduced to:

\begin{align*}
    N_{\text{\tth, \lss}} & = 2 \cdot N_\tth \cdot
    \mathcal{B}(H \rightarrow \tau\tau) \cdot
    \mathcal{B}(W \rightarrow l\nu)^2 \cdot
    \mathcal{B}(W \rightarrow qq)^2                                                            \\
                          & = 2 \cdot 70\ 000 \cdot 0.06 \cdot 0.33^2 \cdot 0.63^2 \approx 363
\end{align*}

Note that $2$ is for two configurations where leptons have either both positive sign, or both negative sign. The first
(positive) case is presented on the \autoref{fig:2lss1tau}, while the second one would have the top quark decay into
$t \rightarrow b\bar{q}q$, antitop quark decay into $\bar{t} \rightarrow \bar{b}l^-\nu$, and then from the Higgs,
$\bar{\tau} \rightarrow \bar{q}q\nu$, and $\tau \rightarrow l^-\nu\nu$.

It is a simplified calculation but it provides a good estimate of the scale of the challenge we face in detecting the
\tth in the \lss channel. As we apply the selection criteria to the simulated data (see \autoref{sec:mc}), we drop from
about 8.9M to just 32K raw events. For the \tth, this corresponds to 0.8M $\rightarrow$ 15K raw events (note that
additional cuts (see \autoref{sec:regions}) are applied, numbers here correspond to the \acrshort{sr}). In terms of
weighted events the transition is from 46K $\rightarrow$ 32.72 weighted events across all the processes, and 523.42
$\rightarrow$ 12.22 weighted events for the \tth. The details for all processes are shown on the
\autoref{tab:class_distributions}.


\subsection{Regions in Particle Physics}
\label{sec:regions}

\begin{figure}[htbp]
    \centering
    \includegraphics[width=0.6\linewidth]{figures/regions.pdf}
    \caption{Relationship between different regions.}
    \label{fig:regions}
\end{figure}

We briefly describe the notion of regions in particle physics. There are three types of regions:

\begin{itemize}
    \item \textbf{\gls{sr}} is where we expect the events of interest, the
          signal, to be most prevalent. It is defined by certain selection criteria that maximize the signal's prominence against
          the background. See the \appref{appendix:cut-expression} for the precise definition of the \gls{sr}.

    \item \textbf{\glspl{cr}} is where we estimate the amount of background contamination
          present in the \gls{sr}. The \gls{cr} is characterized by negligible signal but a significant amount of background
          events, similar to what we expect in the \gls{sr}. By studying the \gls{cr}, we can understand and model the background in the \gls{sr}.

    \item \textbf{\glspl{vr}} are used to test the reliability of our model predictions and the \gls{mc} simulations.
          \glspl{vr} are typically chosen where neither the signal nor the background is expected to be particularly high or
          low. Any significant deviation of the observed data from our model predictions in the \gls{vr} may indicate the
          presence of a new physics process or systematic errors in our model or simulation.
\end{itemize}

These regions are not arbitrarily defined but are carefully chosen based on detailed knowledge of the physics processes
involved and the detector's characteristics.

\begin{figure}[htb]
    \centering
    \begin{subfigure}{0.45\textwidth}
        \includegraphics[width=\linewidth]{figures/yields/lep-pt-0.pdf}
        \caption{Distribution of the transverse momentum of the leading lepton.}
        \label{fig:lep_pt_0}
    \end{subfigure}\hfill%
    \begin{subfigure}{0.45\textwidth}
        \includegraphics[width=\linewidth]{figures/yields/lep-pt-1.pdf}
        \caption{Distribution of the transverse momentum of the subleading lepton.}
        \label{fig:lep_pt_1}
    \end{subfigure}
    \caption[Distributions of the transverse momentum of the leading and subleading leptons inside the Signal Region.]
    {Distributions of the transverse momentum of the leading and subleading leptons inside the \gls{sr}. See
        \autoref{tab:class_distributions} for precise numbers of events of each class. The areas are crossed out because
        the events in these regions are blinded.  One might wonder why the plots are crossed out in some regions. This
        is because the events in these regions are blinded.  Blinding refers to hiding the data points in the bin if the
        relation of teh signal to background is larger than some threshold. A threshold of $0.3$ is used here to
        demonstrate how the unblinding looks in some of the bins. A more common value would be $0.15$. The lower part of
        the plot shows the ratio of the unblinded real data to the \gls{mc} simulation, as well as the statistical
        uncertainty, associated with the bin. \todo{The reasoning behind the blinding is???}.}
\end{figure}

\begin{figure}[htb]
    \centering
    \begin{tabular}{l|rr|rr}
Process & SR (Raw) & SR (Weighted) & ALL (Raw) & ALL (Weighted) \\
\hline
ttH & 15293 & 12.22 & 834970 & 523.42 \\
ttW & 1479 & 5.49 & 581089 & 1680.34 \\
ttW_EW & 74 & 0.59 & 15314 & 122.36 \\
ttZ & 9612 & 7.78 & 1750978 & 1550.66 \\
ttbar & 3 & 0.33 & 299808 & 33487.35 \\
VV & 1433 & 2.55 & 3829589 & 7046.18 \\
tZ & 32 & 0.29 & 40946 & 378.09 \\
WtZ & 209 & 1.20 & 45931 & 248.63 \\
tW & 0 & 0.00 & 4493 & 413.89 \\
threeTop & 449 & 0.16 & 22980 & 8.08 \\
fourTop & 482 & 0.97 & 22746 & 45.27 \\
ggVV & 38 & 0.01 & 792331 & 354.08 \\
VVV & 20 & 0.02 & 151324 & 60.93 \\
VH & 0 & 0.00 & 255 & 102.55 \\
WttW & 94 & 0.81 & 5944 & 50.56 \\
tHjb & 3058 & 0.14 & 445666 & 21.41 \\
tWH & 492 & 0.18 & 30587 & 15.95 \\
\hline
Total & 32768 & 32.72 & 8874951 & 46109.77 \\
\end{tabular}

    \captionof{table}[Comparison of the number of events generated by the MC simulation]{Comparison of the number of events generated by the \gls{mc} simulation (see \autoref{sec:mc})
        before and after applying the \gls{sr} selection criteria. For each process, the number of raw events
        and the number of weighted events are given.}
    \label{tab:class_distributions}
\end{figure}


\autoref{fig:lep_pt_0} and \autoref{fig:lep_pt_1} show the distributions of the leading and
subleading leptons' transverse momenta inside inside the \gls{sr}. These types of plots (yields plots) are one of the
most common plots generated by the \trex\footnote{\trex is a framework for binned template profile likelihood fits
    heavily used at CERN. The documentation can be found at \url{https://trexfitter-docs.web.cern.ch/trexfitter-docs/}.
    The code can be found at \url{https://gitlab.cern.ch/TRExStats/TRExFitter}} software. Such plots are often used to
check the well modelling of the variables, catch some trivial issues early on, and compare the differences between
the different versions of the \gls{mc} production (see \autoref{sec:mc}). More plots are presented in the
\appref{appendix:yields}.




