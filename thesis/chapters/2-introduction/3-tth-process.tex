\section{\tth Process}

The top-antitop and Higgs boson (\tth) process is a specific type of event that occurs within the LHC's high-energy
particle collisions. In this process, two gluons (fundamental particles that mediate the strong force) collide,
resulting in a pair of top quarks and a Higgs boson.

The ttH process is of particular interest in particle physics due to its potential to shed light on the nature of the
Higgs boson. The Higgs boson is pivotal in the Standard Model of particle physics, as it is associated with the Higgs
field, which gives other particles their mass. Despite its significance, the Higgs boson remains one of the least
understood particles due to its elusive nature and the difficulty involved in its detection.

In a ttH event, the Higgs boson quickly decays into other particles, making it impossible to observe directly. Instead,
physicists must reconstruct its presence from the particles into which it decays. The top quarks, on the other hand,
have a higher lifetime, making them easier to detect. The ttH process is unique in that it allows the observation of the
Higgs boson's interactions with top quarks - the heaviest known fundamental particle - thereby providing a direct probe
into the Higgs mechanism.

The primary goal of this research is to separate ttH process events from other events, which is a challenging task given
the similarities between ttH events and certain background processes. By leveraging deep learning techniques, it is
hoped that this task can be achieved more effectively, thereby facilitating a more detailed study of the Higgs boson and
its properties.

