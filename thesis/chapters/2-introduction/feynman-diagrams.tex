\subsection{Feynman Diagrams}

\input{figures/feynman/tth.tex}

A powerful tool in the field of particle physics is the Feynman diagram, named after its creator, the renowned physicist
Richard P. Feynman. Feynman diagrams offer a graphical representation of the mathematical expressions describing the
behavior of subatomic particles. These diagrams have become integral to predicting and understanding the results of
experiments in quantum mechanics, particularly those involving subatomic particles.

In a Feynman diagram, each line represents a particle propagating through space (horizontal axis) over time (vertical
axis). Straight lines depict fermions (matter particles) such as quarks and leptons, while wavy lines represent bosons
(force-carrying particles), including gluons, photons, and $W$ and $Z$ bosons. Interaction vertices, where lines meet,
indicate interactions between particles where energy and momentum are conserved.

Consider, for instance, the \tth process mentioned earlier. In a simple Feynman diagram representing this process, two
incoming gluons (represented as spiral lines) interact at a vertex to produce a top-antitop pair (straight lines) and a
Higgs boson (another straight line). The top and antitop quarks then decay into other particles, typically $W$ bosons and
b-quarks.

These diagrams do not just describe which particles are involved in an interaction; they can also provide insight into
the probability of the interaction occurring. By calculating the areas of the regions enclosed by the lines and vertices
of the diagram, physicists can predict the likelihood of an interaction. However, while Feynman diagrams can be
extremely informative, they also can be complex, with multiple possible diagrams for a single interaction - another
aspect that adds to the challenges faced in particle physics.