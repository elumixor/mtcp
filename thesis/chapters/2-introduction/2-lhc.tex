\section{\gls{lhc} and the \gls{atlas} Detector}

\begin{figure}[h]
    \centering
    \includegraphics[width=\textwidth]{figures/cern.png}
    \caption{\gls{cern}, with the \gls{lhc} and \gls{atlas} detector.}
    \label{fig:cern}
\end{figure}

The \gls{lhc} is the world's largest and highest-energy particle accelerator. It was built by the
\gls{cern} between 1998 and 2008 in collaboration with over 10,000 scientists and
hundreds of universities and laboratories from around the world. The \gls{lhc} is located near Geneva, beneath the border of
Switzerland and France. The accelerator is housed in a tunnel with a circumference of 27 kilometers, buried around 100
meters underground
\footnote{\url{https://home.cern/science/accelerators/large-hadron-collider}}
\footnote{\url{https://home.cern/resources/faqs/facts-and-figures-about-lhc}}.

The \gls{lhc}'s primary purpose is to accelerate protons to near-light speeds and collide them together, thereby producing
extreme conditions that allow us to probe the nature of the subatomic world. The protons are grouped into "bunches" and
circulated in opposite directions around the \gls{lhc} ring. When these bunches cross paths, protons collide, producing
energetic particle showers. The \gls{lhc} is capable of generating up to a billion proton-proton collisions per second.

The \gls{atlas}\footnote{\url{https://home.cern/science/experiments/atlas}} detector is one of the two general-purpose
detectors at the \gls{lhc}. It is a large and complex system designed to measure the properties of particles produced in
the proton-proton collisions. Covering an area equivalent to a five-story building, \gls{atlas} is a multi-layered
device consisting of several subsystems, each optimized to measure different particle properties.

The innermost layer, or the tracking detector, records the trajectory of charged particles, allowing the reconstruction
of their paths. The subsequent layers, calorimeters, measure the energy of particles. Finally, the outermost layer, the
muon spectrometer, specifically designed to detect muons, one of the few particles that can penetrate all inner layers.

The data collected by \gls{atlas} is stored and subsequently analyzed by physicists around the world. The sheer volume of this
data, combined with the complexity of the collision events, presents a significant challenge, requiring sophisticated
statistical methods and computational tools to make sense of the results. Machine learning techniques have been
increasingly employed in recent years to classify and analyze this data, which has led to several significant
breakthroughs in our understanding of particle physics, including the discovery of the Higgs boson in 2012.

