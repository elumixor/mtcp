\glsreset{sm}
\subsection[The Standard Model]{The \acrlong{sm}}

\begin{figure}[htb]
    \centering
    \includegraphics[width=\textwidth]{figures/standard-model.pdf}
    \caption{Particles and forces of the \gls{sm}.}
    \label{fig:standard-model}
\end{figure}

\glsreset{sm}
The framework that describes elementary particles and their interactions is known as the \gls{sm} of particle
physics\footnote{\url{https://home.cern/science/physics/standard-model}}. The \gls{sm} consists of two types of
elementary particles: fermions and bosons. Fermions are particles with half-integer spin and include quarks and leptons,
while bosons are particles with integer spin and act as force carriers
in the \gls{sm}.

Quarks come in six different flavors: up, down, charm, strange, top, and bottom, and interact
through the strong nuclear force. Quarks combine in various ways to form hadrons, such as protons and neutrons.

Leptons, on the other hand, are particles that do not interact via the strong nuclear force. They include electrons,
muons, taus, and their corresponding neutrinos. Each lepton flavor is associated with a specific neutrino.

Bosons are integral to the fundamental forces of the universe. The photon mediates the electromagnetic force, the $W$
and $Z$ bosons mediate the weak nuclear force, and the gluon mediates the strong nuclear force.