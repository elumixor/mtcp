\section{Overview of Particle Physics}

Particle physics, also known as \gls{hep}, is a field of study that seeks to understand the fundamental
constituents of matter and their interactions. This branch of physics has led to the discovery and characterization of a
multitude of subatomic particles, providing deep insights into the structure of the universe.

\subsection{The \gls{sm}}

\begin{figure}[htb]
    \centering
    \includegraphics[width=\textwidth]{figures/standard-model.pdf}
    \caption{The particles and forces of the \gls{sm}.}
    \label{fig:standard-model}
\end{figure}

The framework that describes elementary particles and their interactions is known as the \gls{sm} of particle
physics\footnote{\url{https://home.cern/science/physics/standard-model}}. The \gls{sm} consists of two types of elementary particles: fermions and bosons. Fermions are particles with
half-integer spin and include quarks and leptons, while bosons are particles with integer spin and act as force carriers
in the \gls{sm}.

Quarks come in six different flavors: up, down, charm, strange, top, and bottom, and interact
through the strong nuclear force. Quarks combine in various ways to form hadrons, such as protons and neutrons.

Leptons, on the other hand, are particles that do not interact via the strong nuclear force. They include electrons,
muons, taus, and their corresponding neutrinos. Each lepton flavor is associated with a specific neutrino.

Bosons are integral to the fundamental forces of the universe. The photon mediates the electromagnetic force, the $W$
and $Z$ bosons mediate the weak nuclear force, and the gluon mediates the strong nuclear force.

\subsection{The Higgs Boson}

One additional particle that plays a significant role in the \gls{sm} is the Higgs boson $H$. Unlike the other bosons,
which mediate the fundamental forces, the Higgs boson is associated with the Higgs field, a scalar field that permeates
all of space. The Higgs boson was proposed in 1964 by physicist Peter Higgs and others as a result of their work on the
so-called "Higgs mechanism".

The Higgs mechanism is responsible for the mass of elementary particles. According to this theory, particles acquire
mass by interacting with the Higgs field. The more strongly a particle interacts with this field, the greater its mass.
Particles that do not interact with the Higgs field, such as photons, are massless.

The existence of the Higgs boson and the Higgs field was confirmed experimentally in 2012 by the \gls{atlas} and \gls{cms}
collaborations at \gls{cern}\footnote{\url{https://atlas.cern/updates/feature/higgs-boson}}, a discovery that led to the awarding of the 2013 Nobel Prize in Physics to François Englert and
Peter Higgs.

The Higgs boson itself is unstable and quickly decays into other particles. The decay channels and their corresponding
probabilities, or branching ratios, are predicted by the \gls{sm}. One particular decay of interest in
this work is the Higgs decay into a tau lepton pair, denoted as $H \rightarrow \tau\bar{\tau}$ , which has a branching
ratio of approximately 6\% (\autoref{fig:higgs-decays}).

\begin{figure}[htb]
    \centering
    \includegraphics[width=0.6\textwidth]{figures/higgs-decays.png}
    \caption{The branching ratios for various Higgs boson decay channels. Taken from
        \url{http://opendata.atlas.cern/books/current/get-started/_book/the-higgs-boson.html}}
    \label{fig:higgs-decays}
\end{figure}



\subsection{Particle Interactions and Decays}

In particle physics, interactions between particles result from the exchange of force-carrying particles, or gauge
bosons. Each of the four fundamental forces (gravitational, electromagnetic, strong nuclear, and weak nuclear) has its
associated bosons, which mediate these interactions.

Decays are processes by which a particle transforms into two or more other particles. These are inherently probabilistic
processes, with specific probabilities associated with each possible decay path, given by the branching ratios.


\subsection{Branching Ratios}

A branching ratio (or branching fraction) in particle physics is a measure of the fraction of particles that decay via a
particular decay mode with respect to the total decay modes. It is a critical parameter in understanding particle
interactions, as it gives insight into the relative probabilities of various outcomes of a decay process.

For example, let's consider the production of a top-antitop pair along with a Higgs boson in a proton-proton collision,
also referred to as the \tth process. The Higgs boson can decay into different particles, each with a certain branching
ratio. In the case of the Higgs boson decaying into a bottom-antitop quark pair, the branching ratio is approximately
58\%, which is the highest branching ratio among all the Higgs decay channels.

However, the probability of producing a Higgs boson, along with a top-antitop pair, in a proton-proton collision is very
small, estimated to be about 0.001\% of the total cross-section (\autoref{sec:lss}). This small percentage, coupled with
the fact that the \gls{lhc} produces millions of collisions per second, means that a substantial amount of data must be
analyzed to isolate and identify the relatively few \tth events.

The ability to isolate these rare events depends not only on the branching ratios of the Higgs boson and the top quarks,
but also on our ability to accurately detect and identify the decay products. For instance, in the case of the Higgs
boson decaying into a bottom-antitop pair, we need to be able to identify these 'b-jets' among a plethora of other
particles. This is a major challenge in particle physics and is a central focus of this thesis.

\subsection{Feynman Diagrams}

% ttH
\begin{figure}[htb]
    \centering
    \begin{tikzpicture}
        \begin{feynhand}
            \vertex[dot] (c) at (0,0) {};
            \vertex[dot] (h) at (2,0) {};
            \vertex[dot] (g1) at (-1.5, 1) {};
            \vertex[dot] (g11) at (-3.5,2);
            \vertex[dot] (t) at (2,2);
            \vertex[dot] (g2) at (-1.5, -1) {};
            \vertex[dot] (g21) at (-3.5,-2);
            \vertex[dot] (tbar) at (2,-2);

            % Production of H
            \propag [scalar] (c) to [edge label=\(H\)] (h);
            \propag [anti fermion] (g1) to [edge label=\(\bar{t}\)] (c);
            \propag [fermion] (g2) to [edge label=\(t\)] (c);
            % Production of t
            \propag [fermion] (g1) to [edge label=\(t\)] (t);
            % Production of tbar
            \propag [anti fermion] (g2) to [edge label=\(\bar{t}\)] (tbar);
            % Gluon splitting
            \propag [gluon] (g11) to [edge label=\(g\)] (g1);
            \propag [gluon] (g21) to [edge label=\(g\)] (g2);
        \end{feynhand}
    \end{tikzpicture}
    \caption{Feynman diagram of the \ttH process.}
    \label{fig:feyn_tth}
\end{figure}

A powerful tool in the field of particle physics is the Feynman diagram, named after its creator, the renowned physicist
Richard P. Feynman. Feynman diagrams offer a graphical representation of the mathematical expressions describing the
behavior of subatomic particles. These diagrams have become integral to predicting and understanding the results of
experiments in quantum mechanics, particularly those involving subatomic particles.

In a Feynman diagram, each line represents a particle propagating through space (horizontal axis) over time (vertical
axis). Straight lines depict fermions (matter particles) such as quarks and leptons, while wavy lines represent bosons
(force-carrying particles), including gluons, photons, and $W$ and $Z$ bosons. Interaction vertices, where lines meet,
indicate interactions between particles where energy and momentum are conserved.

Consider, for instance, the \tth process mentioned earlier. In a simple Feynman diagram representing this process, two
incoming gluons (represented as spiral lines) interact at a vertex to produce a top-antitop pair (straight lines) and a
Higgs boson (another straight line). The top and antitop quarks then decay into other particles, typically $W$ bosons and
b-quarks.

These diagrams don't just describe which particles are involved in an interaction; they can also provide insight into
the probability of the interaction occurring. By calculating the areas of the regions enclosed by the lines and vertices
of the diagram, physicists can predict the likelihood of an interaction. However, while Feynman diagrams can be
extremely informative, they also can be complex, with multiple possible diagrams for a single interaction - another
aspect that adds to the challenges faced in particle physics.